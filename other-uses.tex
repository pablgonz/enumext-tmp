% arara: lualatex
% arara: lualatex
% arara: clean: { extensions: [ aux, log, out] }
\documentclass[12pt]{article}
%\documentclass[full]{l3doc}
\usepackage{xcolor}
\usepackage[hyperfootnotes=false]{hyperref}%,footnotehyper ,,footnotehyper
% \hypersetup{}
%\makesavenoteenv{enumext*}
%%\usepackage{multicol}%,footnotehyper ,,footnotehyper
\usepackage{enumext}
%%%\usepackage[base]{babel}
%%%\setlength{\parindent}{20pt}
%%%\setlength{\parskip}{0pt}
\usepackage[top=2cm,bottom=2cm,left=1cm,right=1cm]{geometry}% ,showframe
\usepackage{lua-visual-debug}
% \labelbox
\NewDocumentCommand \itembox { s +m }
  {%
    \IfBooleanTF{#1}
      {\strut\smash{\parbox[t]{\labelwidth}{\raggedright{#2}}}}%
      {\strut\smash{\parbox[t]{\labelwidth}{\raggedleft{#2}}}}%
  }
\newcommand*{\Item}{HOLA }
\begin{document}
Hola


\begin{enumext*}[font=\Huge,itemindent=1cm,columns=2,widest={20}, itemsep=0.1cm,parsep=1cm]
\item Hanza en astillero, \par QUe no quiere nada más en este mundo o el otro. % \anskey(2)*[store-brk]{PPPP}
\item Harna antigua  \color{red} And red!\par QUe no quiere nada más en este mundo o el otro. %\anskey(2)*[store-brk]{PPPP}
\item Hocín flaco, y %\anskey(2)*[store-brk]{PPPP}
\item galgo corredor. %\anskey(2)*[store-brk]{PPPP}
\item Hocín flaco, y %\anskey(2)*[store-brk]{PPPP}
 %%\item
 galgo corredor. %\anskey(2)*[store-brk]{PPPP}
\end{enumext*}

Chao

\begin{enumext}[columns=2,widest={20},itemsep=0.4cm,parsep=0.1cm]
 \item Hanza en astillero,% \par QUe no quiere nada más en este mundo o el otro.  \anskey(2)*[store-brk]{PPPP}
 \item Harna antigua  %\par QUe no quiere nada más en este mundo o el otro. \anskey(2)*[store-brk]{PPPP}
 \item Hocín flaco, y %\anskey(2)*[store-brk]{PPPP}
 \item galgo corredor. %\anskey(2)*[store-brk]{PPPP}
 \item Hocín flaco, y %\anskey(2)*[store-brk]{PPPP}
 \item

 galgo corredor. %\anskey(2)*[store-brk]{PPPP}
  \begin{keyans*} \item hola \end{keyans*}
\end{enumext}


\end{document}

\begin{enumext}[columns=2,widest={20},save-ans=test, itemsep=4cm,parsep=1cm]
 \item* Hanza en astillero, \par QUe no quiere nada más en este mundo o el otro.  \anskey(2)*[store-brk]{PPPP}
 \item Harna antigua  \par QUe no quiere nada más en este mundo o el otro. \anskey(2)*[store-brk]{PPPP}
 \item Hocín flaco, y \anskey(2)*[store-brk]{PPPP}
 \item galgo corredor. \anskey(2)*[store-brk]{PPPP}
 \item(2)* Hocín flaco, y \anskey(2)*[store-brk]{PPPP}
 %%\item
 galgo corredor. \anskey(2)*[store-brk]{PPPP}
\end{enumext}

Below

\end{document}

\stop
\begin{enumext*}[topsep=2cm]
\item \begin{enumext}[topsep=0pt]
    \item A
    \item C \begin{enumerate}[nosep,left=0pt] \item Soy un texto quote \item Soy un texto quote \item Soy un texto quote \end{enumerate}
    \item D \begin{verbatim} ZZ,,,sss \end{verbatim}
 \end{enumext}
\end{enumext*}

\begin{enumerate}[nosep,left=0pt]
  \item Soy un texto quote
    \begin{enumext*}[nosep] \item Soy un texto quote \item Soy un texto quote \item Soy un texto quote \end{enumext*}

  \item Soy un texto quote \end{enumerate}

After
\end{document}
















\begin{enumext}[save-ans=Worksheet,resume,columns=2,nosep,store-ref]%
\item Factor $x^{2}-2x+1$  \anskey{$\left(x-1\right)^{2}$}
\item Factor $3x+3y+3z$    \anskey{$3(x+y+z)$}
\item True False
  \begin{enumext}[no-store,nosep]
    \item $\alpha > \delta$ \anskey{False}
    \item \LaTeX2e\ is cool? \anskey{Very True!}
  \end{enumext}

\item Related to Linux %\anskey{Goooood}
  \begin{enumext}[nosep]
    \item You use linux? \anskey{ZZ Yes}
    \item Usually uses the package manager? \anskey{Yes, dnf}
    %\item Rate the following package and class
      \begin{enumext}
        \item xsim-exam \anskey{doesn't exist for now :(}
        %\item xsim \anskey{very good}
        %\item exsheets  \anskey{obsolete}
      \end{enumext}
  \end{enumext}

\end{enumext}

ANswers

\printkeyans*{Worksheet}



\end{document}







\begin{enumext*}[save-ans=Worksheet,resume,check-ans,columns=2,nosep,store-ref]%
\item Factor $x^{2}-2x+1$  \anskey{$\left(x-1\right)^{2}$}
\item Factor $3x+3y+3z$    \anskey{$3(x+y+z)$}
 \item True False
  \begin{enumext}[no-store]
    \item $\alpha > \delta$ %\anskey{False}
    \item \LaTeX2e\ is cool? %\anskey{Very True!}
  \end{enumext}
\item Related to Linux \anskey{Goooood}
  %\begin{enumext}
    %\item You use linux? \anskey{Yes}
    %\item Usually uses the package manager? \anskey{Yes, dnf}
    %\item Rate the following package and class
      %\begin{enumext}
        %\item xsim-exam \anskey{doesn't exist for now :(}
        %\item xsim \anskey{very good}
        %\item exsheets  \anskey{obsolete}
      %\end{enumext}
  %\end{enumext}

\end{enumext*}

\printkeyans{Worksheet}

\end{document}







%%\printkeyans{test}


\begin{enumext*}[widest={20},resume,nosep,itemsep=0pt]% save-ans=test
 \item Hanza en astillero, \par QUe no quiere nada más en este mundo o el otro.%\strut %\anskey(2)*[store-brk]{PPPP}
 \item Harna antigua \par QUe no quiere nada más en este mundo o el otro.%\strut %\anskey(2)*[store-brk]{PPPP}
 \item Hocín flaco, y %\anskey(2)*[store-brk]{PPPP}
 \item galgo corredor. %\anskey(2)*[store-brk]{PPPP}
 \item Hocín flaco, y %\anskey(2)*[store-brk]{PPPP}
 \item galgo corredor. %\anskey(2)*[store-brk]{PPPP}
\end{enumext*}

%%%\end{document}

Arriba

\begin{enumext*}[itemindent=0pt,columns-sep=20pt,mini-env={3cm}, miniright={HOLgA \\ New}, save-ans=OK, store-ref,show-pos]
 \item* Hanza en astillero, \par QUe no quiere nada más en este mundo o el otro. \anskey(2)*[store-brk]{PPPP}
 \item Harna antigua  \par QUe no quiere nada más en este mundo o el otro.
 \item Hocín flaco, y
 \item galgo corredor.
 \item* Hocín flaco, y
 \item galgo corredor.
\begin{enumext}
\item This text is in the first level. \anskey{ZZZz}
\begin{enumext}
\item This text is in the second level.
\begin{enumext}
\item This text is in the third level.
\begin{enumext}
\item This text is in the fourth level. \anskey{MMMM}
\end{enumext}
\end{enumext}
\end{enumext}
\item[X] This text is in the first level.
\item* This text is in the first level.
\end{enumext}
\end{enumext*}

\smallskip

Abajo \ref{OK:1} and \verb+\ref{OK:2}+ \ref{OK:2} and \verb+\ref{OK:3}+ \ref{OK:3}

%%\printkeyans{OK}
\end{document}

\newpage

ZZZZ

\begin{enumext}[start=1, label=\arabic*.,save-ans=muak,store-ref, show-pos,columns=2,itemsep=10pt,topsep=1cm, ref={\S\arabic*.--}]% ,list-offset=-20pt ,columns-sep=10pt, widest=120 no work
\item(2)* Que aqui $b_n = \frac{(n^3 - 5n)^4}{n^{11}}$ \label{OOO} \anskey{UNO}
\item (3)* AAAA AAA AAAA AA AAA AA AAA\par AAAAAAAAA$c_n = \frac{n^{n + 1}}{n!}$ \anskey*[item-sym*={$\ast$},item-pos*=4pt]{PPPP}

\begin{enumext}
\item This text is in the fourth level. \anskey{MMMM}
\end{enumext}

\item $e_n = \frac{2^{(n^3)}}{n!5^{(n^2)} - n^n}$ \anskey{KKK}
\item*[\S][1cm] A little ZZZZ \anskey{QQQQ}
%%
\item[ZZ] A little not numeber
%%\item!(2) A little paragraph that will be too long to fit on one line no matter what \label{A} ands no more...
%\item*\parbox[t]{\itemwidth}{A little paragraph that will be too long to fit on one line no matter what \label{A} ands no more}
\item A short ZZZ. \anskey{RRR}
\item* $b_n = \frac{(n^3 - 5n)^4}{n^{11}}$ \anskey{LLLL}
\item $c_n = \frac{n^{n + 1}}{n!}$ \anskey{OOOO}
\item* $e_n = \frac{2^{(n^3)}}{n!5^{(n^2)} - n^n}$ \anskey{NNN}
\item A little
    \begin{enumext*}[columns=4]
       \item rocín \anskey{DOS}
       \item rocín \anskey{TRES}
       \item rocín \anskey{CUATRO}
       \item rocín \anskey{ZZZZZZZ}
     \end{enumext*}
%\item $f_n = \sqrt{n + \sqrt{2n}} - \sqrt{n + \sqrt{2n}}x$\footnote{tres}
\item* $e_n = \frac{2^{(n^3)}}{n!5^{(n^2)} - n^n}$ \anskey{NNN}
\end{enumext}

\verb+\ref{muak:3} and \ref{muak:10}+ \ref{muak:3} and \ref{muak:10}

%\end{document}

%%%%%%%%%%%%%%%%%55
\section{PRINT}

\printkeyans{muak}

ARRIBA
\begin{enumext}[topsep=0pt, partopsep=10pt, parsep=20pt]
  \item Uno
  % Aqui necesito calcular \parsep del nivel previo + topsep del nivel actual
  % O simplemente
  \item \leavevmode\vspace{-\dimeval{\baselineskip+\parsep}}

  \begin{enumext*}[before={\setlength{\topsep}{0pt}}, topsep=10pt]
      \item GET
      \item GET
      \item GET
      \item GET
    \end{enumext*}
  \item ZZZ
  \item ZZZ
  \item ZZZ
  \item ZZZ
  \item ZZZ
\end{enumext}


\end{document}


% show-ans,show-pos


\begin{enumext*}[start=1, label=\arabic*.,save-ans=muak,show-ans,show-pos,columns=4,itemsep=10pt,topsep=1cm, ref={\S\arabic*.--}, mini-env=1cm]% ,list-offset=-20pt ,columns-sep=10pt, widest=120 no work
\item(2)* Que aqui $b_n = \frac{(n^3 - 5n)^4}{n^{11}}$ \label{OOO} \anskey{UNO}
\item (3)* AAAA AAA AAAA AA AAA AA AAA\par AAAAAAAAA$c_n = \frac{n^{n + 1}}{n!}$
\item $e_n = \frac{2^{(n^3)}}{n!5^{(n^2)} - n^n}$
\item*[\S][1cm] A little ZZZZ
%%
\item[ZZ] A little not numeber
%%\item!(2) A little paragraph that will be too long to fit on one line no matter what \label{A} ands no more...
%\item*\parbox[t]{\itemwidth}{A little paragraph that will be too long to fit on one line no matter what \label{A} ands no more}
\item A short ZZZ.
\item* $b_n = \frac{(n^3 - 5n)^4}{n^{11}}$
\item $c_n = \frac{n^{n + 1}}{n!}$
\item* $e_n = \frac{2^{(n^3)}}{n!5^{(n^2)} - n^n}$
\item A little
    \begin{enumext}
       \item rocín \anskey{DOS}
       \item rocín \anskey{TRES}
       \item rocín \anskey{CUATRO}
               \begin{enumext} \item rocín \anskey{ZZZZZZZ} \end{enumext}
     \end{enumext}
%\item $f_n = \sqrt{n + \sqrt{2n}} - \sqrt{n + \sqrt{2n}}x$\footnote{tres}
\end{enumext*}

Below  \ref{OOO}

%\getkeyans{muak:2}
\printkeyans{muak} %\ref{muak:2}


\begin{enumext*}[topsep=20pt,columns-sep=20pt,noitemsep,ref={\S\roman*}, itemindent=1cm, mini-env={4cm}]
 \item* lanza en astillero,
 \item adarna antigua
 \item rocín flaco, y
 \item galgo corredor.
        \begin{enumext}[save-ans=muak]
       \item rocín \anskey{DOS}
     \end{enumext}
\end{enumext*}


\end{document}




Arriba

\begin{minipage}[t]{.7\linewidth}

\end{minipage}
\begin{minipage}[t]{.3\linewidth}
 PLOP
\end{minipage}

Abajo

ZZZ
\begin{enumext}[show-length,columns=2,label={\arabic*.},topsep=0pt,labelwidth=1cm,labelsep=0.25cm,ref={\S\arabic*}, wrap-label={\itembox*{#1}},itemindent=1.25cm]%,list-offset=-1cm]
 \item rocín flaco, y
 \item *   \verb+galgo+ corredor XXXXXXXxx aaaaaaaaaaaaa.
 \item[] XXXXXXXxx \item aaaaaaaaaaaaa.
 \item corredor XXXXXXXxx \item aaaaaaaaaaaaa.
 \item corredor XXXXXXXxx \item aaaaaaaaaaaaa.
 \item corredor XXXXXXXxx \item aaaaaaaaaaaaa.
 \item corredor XXXXXXXxx \item aaaaaaaaaaaaa.
\end{enumext} yyy


%%\end{document}
%%%%%
ZZZ \begin{enumext}[columns=3,topsep=0pt,widest={120},noitemsep,ref={\S\arabic*}]%, list-offset=-1cm]
 \item*lanza en astillero,
 \item adarna antigua,\label{Z}
 \item galgo corredor.
\end{enumext} \ref{Z}

ZZZ
\begin{enumext*}[topsep=20pt,columns-sep=20pt,noitemsep,ref={\S\roman*}, itemindent=1cm]

 \item*[\S] lanza en astillero,
 \item adarna antigua,\label{M}
 \item rocín flaco, y
 \item galgo corredor.
\end{enumext*}


ZZZ \ref{M}

\begin{enumext}[columns=2,start=4,topsep=0pt,noitemsep,columns-sep=20pt]
 \item \mbox{}(lanza en astillero,ZZ)
 \item adarna antigua,
 \item rocín flaco, y\footnote{MMMM}
 \item galgo corredor.
 \begin{enumext*}[resume,topsep=0pt,noitemsep,]
 \item* lanza en astillero,
 \item adarna antigua,\footnote{P}
 \item rocín flaco, y
 \item galgo corredor.
\end{enumext*}
\end{enumext} yyy

\ref{M}
En un lugar de la Mancha, de cuyo nombre no quiero acordarme, no ha mucho tiempo que
vivía un hidalgo de los de

\end{document}


ZZZ \begin{enumext*}[topsep=0pt,columns-sep=20pt,itemsep=2pt,ref={\S\arabic*}]
 \item* lanza en astillero,
 \item adarna antigua,\label{J}
 \item rocín flaco, y\footnote{K}
 \item galgo corredor.
 \item lanza en astillero,
  \item galgo corredor.
 \item lanza en astillero,
  \item galgo corredor.
 \item lanza en astillero,
  \item galgo corredor.
 \item lanza en astillero,
 \item galgo corredor.
 \item lanza en astillero,
  \item galgo corredor.
 \item lanza en astillero,
  \item galgo corredor.
 \item Nam dui ligula, fringilla a, euismod
sodales, sollicitudin vel, wisi.
Morbi
auctor lorem non justo. Nam lacus libero,
pretium at, lobortis vitae, ultricies et,
tellus. Donec aliquet, tortor sed accumsan
bibendum, erat ligula aliquet magna,
vitae ornare odio metus a mi. Morbi ac
orci et nisl hendrerit mollis. Suspendisse
ut massa. Cras nec ante. Pellentesque
a nulla. Cum sociis natoque penatibus
et magnis dis parturient montes, nascetur
ridiculus mus.
 \item galgo corredor.
\end{enumext*}
\vfill
\newpage
Esto inicia en 19
 \begin{enumext*}[resume,start=30,widest=123]
 \item lanza en astillero,
 \item galgo corredor.
 \item lanza en astillero,
   \item galgo corredor.
 \item lanza en astillero,
  \item galgo corredor.
 \item lanza en astillero,
  \item galgo corredor.
 \item lanza en astillero,
 \item galgo corredor.
 \item lanza en astillero,
   \item galgo corredor.
 \item lanza en astillero,
  \item galgo corredor.
 \item lanza en astillero,
  \item galgo corredor.
 \item lanza en astillero,
 \item galgo corredor.
 \item lanza en astillero,
  \item galgo corredor.
 \item lanza en astillero,
  \item galgo corredor.
 \item lanza en astillero,
 \item galgo corredor.
 \item lanza en astillero,
  \item galgo corredor.
 \item lanza en astillero,
  \item galgo corredor.
 \item lanza en astillero,
  \item galgo corredor.
 \item lanza en astillero,
\end{enumext*} yyy

\lipsum[2]

\lipsum[2]

\begin{enumext*}[resume]
\item lanza en astillero,
  \item galgo corredor.
 \item lanza en astillero,
  \item galgo corredor.
 \item lanza en astillero,
 \item galgo corredor.
 \item lanza en astillero,
  \item galgo corredor.
 \item lanza en astillero,
  \item galgo corredor.
 \item lanza en astillero,
 \item galgo corredor.
 \item lanza en astillero,
  \item galgo corredor.
 \item lanza en astillero,
  \item galgo corredor.
 \item lanza en astillero,
  \item galgo corredor.
 \item lanza en astillero,
\end{enumext*} yyy








\begin{enumext}[label={},labelsep=0pt,list-indent=0pt,font=\bfseries,noitemsep,topsep=3pt,wrap-label*={\textbullet \bfseries {#1},\enspace}]
 \item[Lo primero que tenía el Quijote] \hfill\\[40pt]lanza en astillero,
 \item[Lo segundo] adarna antigua,
 \item[Lo tercero] rocín flaco, y
 \item[Y por último, lo cuarto] galgo corredor.
\end{enumext}

%\colorbox{lightgray}{\phantom{\rule{\columnwidth-1.5em-\columnsep}{\heightof{1}}}}\\
%

\NewDocumentCommand \stylesat { m }
  {%
    \setlength{\fboxsep}{2pt}%
    \noindent\colorbox{black}{\textcolor{white}{\makebox[\dimeval{\labelwidth-2\fboxsep}][c]{\small\textbf{#1}}}}%
    \colorbox{lightgray}{\phantom{\rule{\linewidth - 2\fboxsep}{\heightof{1}}}}%
  }


\begin{enumext}[list-indent=2cm,listparindent=\dimeval{\itemindent+1cm},start=10,columns=1,nosep,show-length=true]%, wrap-label={\stylesat{#1}}]%
  \item This line use \verb|\item*| A, This line use \verb|\item*| A,

\item B.

  aaaaaaaaaaaaaaaaaa aaaaaaaaaaaaaaaa aaaaaaaa aaaaaaaaaaaaa aaaaaaa
  aaaaaa a aaaaaaaaa aaaaaaaa aaa aaaaaaaaa aaa aa a aaaaaaaa
  aaaaaaaaa aa aaaaaa aaaaaaaa aaaaaaaa a aaaaaaa aaaaaaaaaa aaaaaaaa
  aaaaaaaaaaaaaa.
  %\Item{$\mathcolor{red}{\star}$} This text use %\verb|\Item{hola}|.
  \item[X] This line use \verb|\item[X]|
  \item*[Q1] This text use \verb|\item*[$*$]|. C
  \item   This line use \verb|\item*| A, This line use \verb|\item*| A

  This line use \verb|\item*| A, This line use \verb|\item*| A,\par
  This line use \verb|\item*| A, This line use \verb|\item*| A
  \item This text is in the first level. E
  \begin{enumext}[nosep]
     \item This text is in the second level A
     \item* This line use \verb|\item|*
     \item This text is in the second level C
            \begin{enumext}[,nosep]
               \item This line use \verb|\item|
               \item* This line use \verb|\item*|.
                    \item \begin{enumext}[,nosep]
                            \item* This line use \verb|\item*|
                            \item This line use \verb|\item|
                          \end{enumext}
                  \end{enumext}
        \end{enumext}
  %% \item This text is in the first level.
\end{enumext}


\begin{enumerate}
  \item  This line use \verb|\item*| A
  %\Item{$\mathcolor{red}{\star}$} This text use %\verb|\Item{hola}|.
  \item[] This line use \verb|\item[X]|
  \item*[$*$] This text use \verb|\item*[$*$]|. C
  \item This text is in the first level. D
  \item This text is in the first level. E
\end{enumerate}








 \newlength{\desitemcwd}
 \settowidth{\desitemcwd}{\textbf{Something long}}


 \begin{enumext}[label={},labelsep=4pt,labelwidth=\desitemcwd,list-offset=-\dimeval{\labelwidth+\labelsep}, font=\bfseries,noitemsep,topsep=3pt]
  \item[SomeThing] A short one-line description.
  \item This is an entry \textit{without} a label.
  \item[Something] A short one-line description.
  \item[Something long] A much longer description. Lorem ipsum dolor sit amet, consectetuer adipiscing elit.
    Ut purus elit, vestibulum ut, placerat ac, adipiscing vitae, felis.
    Curabitur dictum gravida mauris.
 \end{enumext}






 \begin{enumext}[label={},labelsep=4pt,labelwidth=\desitemcwd,list-offset=-\dimeval{\desitemcwd+4pt}, font=\bfseries,noitemsep,topsep=3pt]
  \item[SomeThing] A short one-line description.
  \item This is an entry \textit{without} a label.
  \item[Something] A short one-line description.
  \item[Something long] A much longer description. Lorem ipsum dolor sit amet, consectetuer adipiscing elit.
    Ut purus elit, vestibulum ut, placerat ac, adipiscing vitae, felis.
    Curabitur dictum gravida mauris.
 \end{enumext}


 \begin{enumext}[label={},align=right,labelsep=4pt,labelwidth=\desitemcwd,list-offset=-\dimeval{\desitemcwd+4pt}, font=\bfseries,noitemsep,topsep=3pt]
  \item[SomeThing] A short one-line description.
  \item This is an entry \textit{without} a label.
  \item[Something] A short one-line description.
  \item[Something long] A much longer description. Lorem ipsum dolor sit amet, consectetuer adipiscing elit.
    Ut purus elit, vestibulum ut, placerat ac, adipiscing vitae, felis.
    Curabitur dictum gravida mauris.
 \end{enumext}
%

\end{document}










\Item{ss}

\begin{enumext}[list-offset=-30pt, start=10,widest=10,nosep,labelsep=1cm]%,wrap-label={\emph{#1}}]
  \item*  This line use \verb|\item*| A
  %\Item{$\mathcolor{red}{\star}$} This text use %\verb|\Item{hola}|.
  \item[X] This line use \verb|\item[X]|
  \item*[$*$] This text use \verb|\item*[$*$]|. C
  \item This text is in the first level. D
  \item This text is in the first level. E
  \begin{enumext}[nosep]
     \item This text is in the second level A
     \item* This line use \verb|\item|*
     \item This text is in the second level C
            \begin{enumext}[,nosep]
               \item This line use \verb|\item|
               \item* This line use \verb|\item*|.
                    \item \begin{enumext}[,nosep]
                            \item* This line use \verb|\item*|
                            \item This line use \verb|\item|
                          \end{enumext}
                  \end{enumext}
        \end{enumext}
   \item This text is in the first level.
\end{enumext}

See \ref{star}

\Item{ss}
\end{document}

\begin{enumext}[labelwidth=20pt,start=1]%,wrap-label={\emph{#1}}]
                            \item This text is in the fourth level. \label{level-4}
                            \item This text is in the fourth level.
\end{enumext}






 \newlength{\desitemcwd}
 \settowidth{\desitemcwd}{\textbf{Something long}}
 \begin{enumext}[label={},labelwidth=\desitemcwd,font=\bfseries,noitemsep,topsep=3pt,wrap-label={\itembox*{#1}}]
  \item This is an entry \textit{without} a label.
  \item[Something] A short one-line description.
  \item[Something long] A much longer description. orem ipsum dolor sit amet, consectetuer adipiscing elit.
    Ut purus elit, vestibulum ut, placerat ac, adipiscing vitae, felis.
    Curabitur dictum gravida mauris. Nam arcu libero, nonummy
    eget, consectetuer id, vulputate a, magna. Donec vehicula augue
 \end{enumext}

 \subsubsection*{Description indented by label in the left margin}

%\dimeval{-\labelwidth -\labelsep}% \dimexpr -\labelwidth -\labelsep
 \begin{enumext}[label={},labelsep=4pt,labelwidth=\desitemcwd,list-offset=-\dimeval{\desitemcwd + 4pt},font=\bfseries,,wrap-label={\itembox{#1}},noitemsep,topsep=3pt]%  align=right,
  \item This is an entry \textit{without} a label.
  \item[Something] A short one-line description.
  \item[Something \\ long] A much longer description. orem ipsum dolor sit amet, consectetuer adipiscing elit.
    Ut purus elit, vestibulum ut, placerat ac, adipiscing vitae, felis.
    Curabitur dictum gravida mauris. Nam arcu libero, nonummy
    eget, consectetuer id, vulputate a, magna. Donec vehicula augue
 \end{enumext}
This is

 \begin{enumext}[label={},labelsep=4pt,labelwidth=\dimeval{\desitemcwd},list-offset={\dimeval{-\labelwidth - 4pt}}, align=right,font=\bfseries,noitemsep,topsep=3pt]
  \item This is an entry \textit{without} a label.
  \item[Something] A short one-line description.
  \item[Something long] A much longer description. orem ipsum dolor sit amet, consectetuer adipiscing elit.
    Ut purus elit, vestibulum ut, placerat ac, adipiscing vitae, felis.
    Curabitur dictum gravida mauris. Nam arcu libero, nonummy
    eget, consectetuer id, vulputate a, magna. Donec vehicula augue
 \end{enumext}
This is
\end{document}


\subsubsection*{Fake itemize}
\noindent\begin{minipage}[t]{0.45\linewidth}
\begin{enumext}[nosep, label=\textbullet]
  \item First level item
    \begin{enumext}[nosep,label=\normalfont\bfseries\textendash]
      \item Second level item
        \begin{enumext}[nosep,label=\textasteriskcentered]
          \item Third level item
            \begin{enumext}[nosep,label=\textperiodcentered]
                \item Fourth level item
            \end{enumext}
        \end{enumext}
    \end{enumext}
  \item First level item
\end{enumext}
\end{minipage}\hfill
\begin{minipage}[t]{0.45\linewidth}
\begin{enumext}[nosep, label={$\triangleright$}]
  \item First level item
    \begin{enumext}[nosep,label=$\diamond$]
      \item Second level item
        \begin{enumext}[nosep,label=$\circ$]
          \item Third level item
            \begin{enumext}[nosep,label=$\star$]
                \item Fourth level item
            \end{enumext}
        \end{enumext}
    \end{enumext}
  \item First level item
\end{enumext}
\end{minipage}

\subsubsection*{Fake description}

\subsubsection*{Standart with \texttt{list-indent=2.5em}}
\begin{enumext}[label={},list-indent=2.5em,font=\bfseries,nosep]
  \item \hspace{-\labelsep}This is an entry \textit{without} a label.
  \item[Something short] A short one-line description.
  \item[Something long long] A much longer description. orem ipsum dolor sit amet, consectetuer adipiscing elit.
    Ut purus elit, vestibulum ut, placerat ac, adipiscing vitae, felis.
    Curabitur dictum gravida mauris. Nam arcu libero, nonummy
    eget, consectetuer id, vulputate a, magna. Donec vehicula augue
\end{enumext}

\subsubsection*{Widest with \texttt{list-indent=0pt}}
\begin{enumext}[label={},list-indent=0pt,font=\bfseries,nosep]
  \item \hspace{-\labelsep}This is an entry \textit{without} a label.
  \item[Something short] A short one-line description.
  \item[Something long long] A much longer description. orem ipsum dolor sit amet, consectetuer adipiscing elit.
    Ut purus elit, vestibulum ut, placerat ac, adipiscing vitae, felis.
    Curabitur dictum gravida mauris. Nam arcu libero, nonummy
    eget, consectetuer id, vulputate a, magna. Donec vehicula augue
\end{enumext}

\subsubsection*{Indent whit label \texttt{labelwidth=descritp-width}}
\newlength\descrwidth
\settowidth{\descrwidth}{\textbf{Something long long}}
\begin{enumext}[label={},labelwidth=\descrwidth,font=\bfseries,nosep]
  \item This is an entry \textit{without} a label.
  \item[Something short] A short one-line description.
  \item[Something long long] A much longer description. orem ipsum dolor sit amet, consectetuer adipiscing elit.
    Ut purus elit, vestibulum ut, placerat ac, adipiscing vitae, felis.
    Curabitur dictum gravida mauris. Nam arcu libero, nonummy
    eget, consectetuer id, vulputate a, magna. Donec vehicula augue
\end{enumext}

\subsubsection*{Indent whit label \texttt{labelwidth=descritp-width,list-indent=20pt}}
\begin{enumext}[label={},align=left,labelwidth=\descrwidth,list-indent=20pt,font=\bfseries,nosep]
  \item This is an entry \textit{without} a label.
  \item[Something] A short one-line description.
  \item[Something long] A much longer description. orem ipsum dolor sit amet, consectetuer adipiscing elit.
    Ut purus elit, vestibulum ut, placerat ac, adipiscing vitae, felis.
    Curabitur dictum gravida mauris. Nam arcu libero, nonummy
    eget, consectetuer id, vulputate a, magna. Donec vehicula augue
\end{enumext}

\subsubsection*{Description traslated to left margin}

\begin{enumext}[label={},align=right,labelsep=5pt,labelwidth=\descrwidth,list-offset={\dimexpr -\descrwidth -\labelsep},font=\bfseries,nosep]
  \item This is an entry \textit{without} a label.
  \item[Something] A short one-line description.
  \item[Something long] A much longer description. orem ipsum dolor sit amet, consectetuer adipiscing elit.
    Ut purus elit, vestibulum ut, placerat ac, adipiscing vitae, felis.
    Curabitur dictum gravida mauris. Nam arcu libero, nonummy
    eget, consectetuer id, vulputate a, magna. Donec vehicula augue
\end{enumext}

\subsubsection*{Description traslated to left margin whit parbox}
%\NewDocumentCommand\paritem{+m}{\strut\smash{\parbox[t]{\labelwidth}{\raggedright{#1}}}}
\NewDocumentCommand\parleft{+m}{\strut\smash{\parbox[t]{\labelwidth}{\raggedright{#1}}}}
\NewDocumentCommand\parnoleft{+m}{\strut\smash{\parbox[t]{\labelwidth}{\raggedleft{#1}}}}

\begin{enumext}[label={},labelsep=5pt,labelwidth=\descrwidth,list-offset={\dimexpr -\descrwidth -\labelsep},font=\bfseries,nosep, wrap-label={\parnoleft{#1}}]
  \item This is an entry \textit{without} a label.
  \item[Something] A short one-line description.
  \item[Something \\ long] A much longer description. orem ipsum dolor sit amet, consectetuer adipiscing elit.
    Ut purus elit, vestibulum ut, placerat ac, adipiscing vitae, felis.
    Curabitur dictum gravida mauris. Nam arcu libero, nonummy
    eget, consectetuer id, vulputate a, magna. Donec vehicula augue
\end{enumext}

\subsubsection*{UNProfesional description NOT traslated whit parbox}

\begin{enumext}[label={},labelsep=5pt,labelwidth=\descrwidth,font=\bfseries,nosep, wrap-label={\parnoleft{#1}}]
  \item This is an entry \textit{without} a label.
  \item[Something] A short one-line description.
  \item[Something \\ long] A much longer description. orem ipsum dolor sit amet, consectetuer adipiscing elit.
    Ut purus elit, vestibulum ut, placerat ac, adipiscing vitae, felis.
    Curabitur dictum gravida mauris. Nam arcu libero, nonummy
    eget, consectetuer id, vulputate a, magna. Donec vehicula augue
\end{enumext}


\subsubsection*{Test using hooks for item command}

\begin{enumext}[label={},labelwidth=3em,labelsep=5pt,font=\bfseries,nosep]
  \item \hfill\\ This is an entry \textit{without} a label.
  \item[Something] \hfill\\ A short one-line description.
  \item[Something \\ long] A much longer description. orem ipsum dolor sit amet, consectetuer adipiscing elit.
    Ut purus elit, vestibulum ut, placerat ac, adipiscing vitae, felis.
    Curabitur dictum gravida mauris. Nam arcu libero, nonummy
    eget, consectetuer id, vulputate a, magna. Donec vehicula augue
\end{enumext}
\end{document}




\begin{description}
   \item This is an entry \textit{without} a label.
   \item[Something short] A short one-line description.
   \item[Something long] A much longer description. orem ipsum dolor sit amet, consectetuer adipiscing elit.
    Ut purus elit, vestibulum ut, placerat ac, adipiscing vitae, felis.
    Curabitur dictum gravida mauris. Nam arcu libero, nonummy
    eget, consectetuer id, vulputate a, magna. Donec vehicula augue
\end{description}

\begin{description}
  \item[Fedora 6] orem ipsum dolor sit amet, consectetuer adipiscing elit.
    Ut purus elit, vestibulum ut, placerat ac, adipiscing vitae, felis.
    Curabitur dictum gravida mauris. Nam arcu libero, nonummy
    eget, consectetuer id, vulputate a, magna. Donec vehicula augue

    eu neque. Pellentesque habitant morbi tristique senectus et netus
    et malesuada fames ac turpis egestas. Mauris ut leo
  \item[Fedora 8] Code name Werewolf
\end{description}


\begin{enumext}[labelwidth=4.5em, labelsep=0.5em,list-offset=-5em,font=\bfseries,mini-env={0.4\linewidth}]
  \item[Fedora 2023] orem ipsum dolor sit amet, consectetuer adipiscing elit.
    Ut purus elit, vestibulum ut, placerat ac, adipiscing vitae, felis.
    Curabitur dictum gravida mauris. Nam arcu libero, nonummy
    eget, consectetuer id, vulputate a, magna. Donec vehicula augue

    eu neque. Pellentesque habitant morbi tristique senectus et netus
    et malesuada fames ac turpis egestas. Mauris ut leo
  \item[Fedora 8] Code name Werewolf
      \miniright
    (a) P here
\end{enumext}




\section{Unumbered list}
%% Definition of |\miniright| command.
\NewDocumentCommand \parleft { +m }
  {
    \strut\smash{\parbox[t]{\labelwidth}{\raggedright{#1}}}
  }

% label={},
%\begin{enumext}[labelwidth=8em, list-offset=1cm,columns=1,wrap-label={\strut\smash{\parbox[t]{\labelwidth}{\raggedright{#1}}}},nosep]
\begin{enumext}[labelwidth=8em, list-indent=2cm,columns=1,wrap-label={\parleft{#1}},nosep]
  \item[First \\ Key \\ in ] orem ipsum dolor sit amet, consectetuer adipiscing elit.
Ut purus elit, vestibulum ut, placerat ac, adipiscing vitae, felis.
Curabitur dictum gravida mauris. Nam arcu libero, nonummy
eget, consectetuer id, vulputate a, magna. Donec vehicula augue
eu neque. Pellentesque habitant morbi tristique senectus et netus
et malesuada fames ac turpis egestas. Mauris ut leo
  \item This text is in the first level
  \item \begin{enumext}
          \item This text is in the second level. \label{level-2}
            \item \begin{enumext}
                    \item This text is in the third level. \label{level-3}
                    \item \begin{enumext}
                            \item This text is in the fourth level. \label{level-4}
                            \item This text is in the fourth level.
                          \end{enumext}
                  \end{enumext}
        \end{enumext}
   \item This text is in the first level.
\end{enumext}

XXX

\begin{enumext}[list-indent=0pt,align=left,labelsep=0cm,list-offset=1cm, rightmargin=1cm,font=\normalfont,first={\em}]%,,wrap-label={\emph{#1}}]
  \item orem ipsum dolor sit amet, consectetuer adipiscing elit.
Ut purus elit, vestibulum ut, placerat ac, adipiscing vitae, felis.
Curabitur dictum gravida mauris. Nam arcu libero, nonummy
eget, consectetuer id, vulputate a, magna. Donec vehicula augue
eu neque. Pellentesque habitant morbi tristique senectus et netus
et malesuada fames ac turpis egestas. Mauris ut leo
  \item ipsum dolor sit amet, consectetuer adipiscing elit.
Ut purus elit, vestibulum ut, placerat ac, adipiscing vitae, felis.
Curabitur dictum gravida mauris. Nam arcu libero, nonummy
eget, consectetuer id, vulputate a, magna. Donec vehicula augue
eu neque. Pellentesque habitant morbi tristique senectus et netus
et malesuada fames ac turpis egestas. Mauris ut leo
  \item orem ipsum dolor sit amet, consectetuer adipiscing elit.
Ut purus elit, vestibulum ut, placerat ac, adipiscing vitae, felis.
Curabitur dictum gravida mauris. Nam arcu libero, nonummy
eget, consectetuer id, vulputate a, magna. Donec vehicula augue
eu neque. Pellentesque habitant morbi tristique senectus et netus
et malesuada fames ac turpis egestas. Mauris ut leo
  \item \begin{enumext}[first=\normalfont]
          \item This text is in the second level.
            \item \begin{enumext}
                    \item This text is in the third level.
                          \item \begin{enumext}
                            \item This text is in the fourth level.
                            \item This text is in the fourth level.
                          \end{enumext}
                  \end{enumext}
        \end{enumext}
   \item This text is in the first level.
\end{enumext}


\end{document}


\section{Test label and ref}

\begin{enumext}[label={\em\alph*)}]%,wrap-label={\emph{#1}}]
  \item This text is in the first level \label{level-1}.
  \item \begin{enumext}
          \item This text is in the second level. \label{level-2}
            \item \begin{enumext}
                    \item This text is in the third level. \label{level-3}
                    \item \begin{enumext}
                            \item This text is in the fourth level. \label{level-4}
                            \item This text is in the fourth level.
                          \end{enumext}
                  \end{enumext}
        \end{enumext}
   \item This text is in the first level.
\end{enumext}


Ref to |level-1| \ref{level-1}, Ref to |level-2| \ref{level-2}, Ref to |level-3| \ref{level-3}, Ref to |level-4| \ref{level-4},

\section{Test widest}

\begin{enumext}[list-indent=0pt,align=right,labelwidth=1cm,labelsep=0cm]%,,wrap-label={\emph{#1}}]
  \item \lipsum[1-2] \par TEXTXXXXXXXXXXXXXXXXXXXXXXXXXX
  \item \lipsum[1-2]
  \item \lipsum[1-2][3-4]
  \item \begin{enumext}
          \item This text is in the second level.
            \item \begin{enumext}
                    \item This text is in the third level.
                          \item \begin{enumext}
                            \item This text is in the fourth level.
                            \item This text is in the fourth level.
                          \end{enumext}
                  \end{enumext}
        \end{enumext}
   \item This text is in the first level.
\end{enumext}

\end{document}



%<*example>
% \fi
\begin{examplecode}[frame=single]
\NewDocumentCommand \parleft { +m }
  {
    \strut\smash{\parbox[t]{\labelwidth}{\raggedright{#1}}}
  }
\end{examplecode}
% \iffalse
%</example>







Here\footnote{First type of footnotes.} is some text with three

And a footnote with some verbatim material%
\begin{footnote}
The footnote environment allows verbatim contents: \verb|&$^%\[}$|
\end{footnote}.

And \footnotemark[3] more


\footnotetext[3]{Some text}

\sepfootnotecontent{A}{This is a more convenient place to
  code the footnote text.}

Here\sepfootnote{A} is some text.
\newpage















\documentclass{article}
\begin{document}
\ExplSyntaxOn
\tl_set:Nn \l_tmpa_tl { @@ }
\regex_replace_once:nnN { @{2} }{ enumext }  \l_tmpa_tl
 \l_tmpa_tl
\ExplSyntaxOff
\end{document}



\NewDocumentCommand \parleft { +m }
{
  \begin{pgfpicture}{-1ex}{-0.65ex}{1ex}{1ex}
    {
      %\pgftransformshift{\pgfpoint{0pt}{0.5pt}}
      \pgftext{#1}
     }
  \end{pgfpicture}%
}

\begin{frame}{Unordered Lists in Beamer}
Some text % label={$\blacktriangleright$},
\begin{enumext}[columns=1,topsep=0pt,font={\usebeamerfont{enumerate item}\usebeamercolor[fg]{enumerate item}},mini-env={0.3\linewidth}]%wrap-label={{#1}}]
  \item  orem ipsum dolor sit amet, consectetuer adipiscing elit.
  \item \begin{enumext}[font={\usebeamerfont{enumerate item}\usebeamercolor[fg]{enumerate item}}]%[label=\normalfont\bfseries\textendash]
          \item This text is in the second level. \label{level-2}
            \item \begin{enumext}[font={\usebeamerfont{enumerate item}\usebeamercolor[fg]{enumerate item}}]%[label=\textasteriskcentered]
                    \item This text is in the third level. \label{level-3}
                    \item \begin{enumext}[font={\usebeamerfont{enumerate item}\usebeamercolor[fg]{enumerate item}}]%[label=\textperiodcentered]
                            \item This text is in the fourth level. \label{level-4}
                            \item This text is in the fourth level.
                          \end{enumext}
                  \end{enumext}
        \end{enumext}
   \item This text is in the first level.
       \miniright
    (a) P here
\end{enumext}

\end{frame}

  \begin{frame}
\frametitle{explanation}
\begin{columns}
\begin{column}{0.5\textwidth}
   some text here some text here some text here some text here some text here
\end{column}
\begin{column}{0.5\textwidth}
    \begin{center}
     %%%%% this is a minipage, so \textwidth is already adjusted to the size of the column
     \includegraphics[width=\textwidth]{image1.jpg}
     \end{center}
\end{column}
\end{columns}
\end{frame}

\end{document}

% \subsubsection*{Example 5}
%
% \iffalse
%<*example>
% \fi
\begin{filecontents*}[overwrite]{enumext-exa-5.tex}
% arara: pdflatex
% arara: clean: { extensions: [ aux, log] }
\documentclass{article}
\usepackage[top=2cm,bottom=2cm,left=2cm,right=2cm]{geometry}%
\usepackage{xcolor}
\usepackage{enumext}
\newlength{\satruleht}
\settoheight{\satruleht}{\small\textbf{1}}
\NewDocumentCommand \stylesat { m }
  {%
    \setlength{\fboxsep}{2pt}%
    \noindent\colorbox{black}{\textcolor{white}{\makebox[\dimeval{\labelwidth-2\fboxsep}][c]{\small\textbf{#1}}}}%
    \colorbox{lightgray}{\phantom{\rule{\dimeval{\linewidth-2\fboxsep}}{\satruleht}}}%
  }
\pagestyle{empty}
\begin{document}
\setenumext[keyans]{label=\Alph*),font=\small,nosep}
\begin{enumext}[label=\arabic*,labelwidth=20pt,labelsep=0pt,columns=2,save-ans=stylesat, wrap-label={\stylesat{#1}}]
 \item Which choice best describes what happens in the passage?
   \begin{keyans}
    \item* One character argues with another character who intrudes on her home.
    \item One character receives a surprising request from another character.
    \item One character reminisces about choices she has made over the years.
    \item One character criticizes another character for pursuing an unexpected course of action.
   \end{keyans}

 \item Which choice best describes what happens in the passage?
   \begin{keyans}
    \item One character argues with another character who intrudes on her home.
    \item One character receives a surprising request from another character.
    \item* One character reminisces about choices she has made over the years.
    \item One character criticizes another character for pursuing an unexpected course of action.
   \end{keyans}

 \item Which choice best describes what happens in the passage?
   \begin{keyans}
    \item One character argues with another character who intrudes on her home.
    \item* One character receives a surprising request from another character.
    \item One character reminisces about choices she has made over the years.
    \item One character criticizes another character for pursuing an unexpected course of action.
   \end{keyans}

 \item Which choice best describes what happens in the passage?
   \begin{keyans}
    \item One character argues with another character who intrudes on her home.
    \item One character receives a surprising request from another character.
    \item One character reminisces about choices she has made over the years.
    \item* One character criticizes another character for pursuing an unexpected course of action.
   \end{keyans}
\end{enumext}

\printkeyans[columns=4]{stylesat}
\end{document}
\end{filecontents*}
% \iffalse
%</example>
% \fi
%
% Adapted from the response given by Stephen in \href{SAT like question format}{https://tex.stackexchange.com/a/691544}
% \textattachfile[color=red,print=false]{enumext-exa-5.tex}{\faFile*[regular]}.
% \setenumext[keyans]{label=\Alph*),font=\small,nosep}
% \begin{enumext}[label=\arabic*,labelwidth=20pt,labelsep=0pt,columns=2,save-ans=stylesat, wrap-label={\stylesat{#1}}]
%  \item Which choice best describes what happens in the passage?
%   \begin{keyans}
%     \item* One character argues with another character who intrudes on her home.
%     \item One character receives a surprising request from another character.
%     \item One character reminisces about choices she has made over the years.
%     \item One character criticizes another character for pursuing an unexpected course of action.
%   \end{keyans}
%
%  \item Which choice best describes what happens in the passage?
%   \begin{keyans}
%    \item One character argues with another character who intrudes on her home.
%    \item One character receives a surprising request from another character.
%    \item* One character reminisces about choices she has made over the years.
%    \item One character criticizes another character for pursuing an unexpected course of action.
%   \end{keyans}
%
%  \item Which choice best describes what happens in the passage?
%   \begin{keyans}
%    \item One character argues with another character who intrudes on her home.
%    \item* One character receives a surprising request from another character.
%    \item One character reminisces about choices she has made over the years.
%    \item One character criticizes another character for pursuing an unexpected course of action.
%   \end{keyans}
%
%  \item Which choice best describes what happens in the passage?
%   \begin{keyans}
%    \item One character argues with another character who intrudes on her home.
%    \item One character receives a surprising request from another character.
%    \item One character reminisces about choices she has made over the years.
%    \item* One character criticizes another character for pursuing an unexpected course of action.
%   \end{keyans}
% \end{enumext}
%
% \printkeyans[columns=4]{stylesat}
