% arara: lualatex
% arara: lualatex
% arara: clean: { extensions: [ aux, log, out] }
\documentclass[10pt]{article}
\usepackage{enumext,tikzducks} % ducks :-)
%\usepackage{hyperref}%,,hyperref
\usepackage{cleveref}
\usepackage{footnotehyper}%,footnotehyper ,
%%\def\FNH@prefntext{\@makefntext{}}
\makesavenoteenv{enumext}
\makesavenoteenv{enumext*}
\usepackage[top=2cm,bottom=2cm,left=1cm,right=1cm,showframe]{geometry}
%\usepackage[showframe]{geometry}
%\setlength{\parskip}{50pt}
\setlength{\parindent}{0pt}
\setlength{\columnseprule}{0.5pt}
\usepackage{lua-visual-debug}
% Revisar la implementacion cerca de las linas 830 ; 1628

%\usepackage{cleveref}

% Create an alias for 'item' (of enumi level):
% https://tex.stackexchange.com/questions/553519/local-name-change-of-enumi-with-cleveref
% \crefalias{postulate}{enumXi}
% \crefname{postulate}{postulate}{postulates}
% \Crefname{postulate}{Postulate}{Postulates}
\crefname{enumXi}{postulate}{postulates}
\Crefname{enumXi}{Postulate}{Postulates}

\AtBeginEnvironment{tikzpicture}{\tracinglostchars=0\relax}
\begin{document}

BEFORE XXX
\begin{enumext}[save-ans=muaksss,columns=2,columns-sep=20pt,itemindent=0.5cm,listparindent=0.5cm,parsep=2\baselineskip]%[topsep=0pt,parsep=0pt,partopsep=0pt,itemsep=0pt,columns=1]%,,itemsep=0pt

\item* Using [columns=1] here UNO
Vertical spacing should be kept constant on all interior levels (or
appear to be constant). Changing the font size or line spacing should
not affect the output (much).%\anskey{mmm}


  \begin{enumext}[nosep]%,topsep=10pt,parsep=0pt,partopsep=0cm
     \item* P\anskey{mmm} \item Q\anskey{mmm}% \item Q \anskey{mmm} \item R\anskey{mmm} \item S\anskey{mmm}\footnote{quak} \anskey{mmm}
  \end{enumext}

\item Using [columns=2] here below \anskey{mmm}

%\begin{enumext}[nosep]%,topsep=20pt,parsep=10pt
     %\item  P \item Q \item X  \item R \item S  \anskey{mmm}
  %\end{enumext}

%%%\columnbreak

%\item Using [columns=2,mini-env] here DOS

  %\begin{enumext}[columns*=5,mini-env=0.4\linewidth,nosep]%,topsep=10pt,parsep=30pt
    %\item  P \item Q \item RS \item T \item U\footnote{ditto} \anskey{mmm}
    %\miniright
    %\rule{30pt}{20pt}\par
    %P Tikz image here
  %\end{enumext}

%\item Using [columns=1] here below

%\begin{enumext}[nosep]%,topsep=20pt,parsep=10pt
     %\item  P \item Q \item X  \item R \item S\footnote{Otto} \anskey{mmm}
  %\end{enumext}

\end{enumext}

After enumext XXXXXXXXXXXXXXXXXXXX

\printkeyans{muaksss}


%%\end{document}



BEFORE XXX in horizontal mode

\begin{enumext}[columns=2,nosep,label={(\arabic*.)}]%[topsep=0pt,parsep=0pt,partopsep=0pt,itemsep=0pt,columns=1]%,,itemsep=0pt

   \item\label{1} Blah % note use of optional argument of \label
   \item\label{2} Blah % ditto
   \item\label{3} Blah

\item Using [columns=1,nosep] here UNO \label{test}

  \begin{enumext}[columns=1,nosep,label=\arabic{enumXi}.(\alph*)]%,topsep=10pt,parsep=0pt,partopsep=0cm
     \item  P \item Q \item Q
   %\item R \item S
  \end{enumext}

\item Using [columns=2,nosep] here below
\begin{enumext}[columns=1,nosep]%,topsep=20pt,parsep=10pt
     \item  P \item Q \item X  \item R \item S
  \end{enumext}

\columnbreak

\item Using [columns=2,nosep] here DOS
  \begin{enumext}[columns=2,nosep]%,topsep=10pt,parsep=30pt
    \item  P \item Q \item RS \item T \item U
  \end{enumext}

\item Using [columns=1,nosep] here below
\begin{enumext}[columns=1,nosep]%,topsep=20pt,parsep=10pt
     \item  P \item Q \item X  \item R \item S
  \end{enumext}

\end{enumext}


\newpage

After enumext [nosep] in vertical mode \ref{test}. See \cref{2,3}. \Cref{1} instead, perhaps.

BEFORE XXX in verical mode

\begin{enumext}[columns=2,nosep,resume]%[topsep=0pt,parsep=0pt,partopsep=0pt,itemsep=0pt,columns=1]%,,itemsep=0pt

\item Using [columns=1] here UNO \label{MM}

  \begin{enumext}[columns=1,nosep]%,topsep=10pt,parsep=0pt,partopsep=0cm
     \item  P \item Q \item Q
   %\item R \item S
  \end{enumext}

\item Using [columns=2] here below

\begin{enumext}[columns=1,nosep]%,topsep=20pt,parsep=10pt
     \item  P \item Q \item X  \item R \item S
  \end{enumext}

\columnbreak

\item Using [columns=2] here DOS

  \begin{enumext}[columns=2,nosep]%,topsep=10pt,parsep=30pt
    \item  P \item Q \item RS \item T \item U
  \end{enumext}

\item Using [columns=1] here below

\begin{enumext}[columns=1,nosep]%,topsep=20pt,parsep=10pt
     \item  P \item Q \item X  \item R \item S
  \end{enumext}

\end{enumext}

After enumext \ref{MM}


TEST mini-env - MULTICOLS
% Following manually tweaked to show the problem.
\begin{tikzpicture}[overlay, remember picture]
    \draw [line width=0.6pt,red]  (0.5cm,-1.535cm) -- (19.5cm,-1.535cm);
    \draw [line width=2.6pt,blue] (0.5cm,-3.749cm) -- (19.5cm,-3.749cm);
    \draw [line width=0.6pt,orange] (0.5cm,-2.749cm) -- (19.5cm,-2.749cm);
\end{tikzpicture}


BEFORE XXX

\begin{enumext}[columns=2,nosep]%[topsep=0pt,parsep=0pt,partopsep=0pt,itemsep=0pt,columns=1]%,,itemsep=0pt

\item Using [columns=1] here UNO

  \begin{enumext}[columns=1,nosep]%,topsep=10pt,parsep=0pt,partopsep=0cm
     \item  P \item Q \item Q  \item R \item S
  \end{enumext}

\item Using [columns=2] here below

\begin{enumext}[columns=1,nosep]%,topsep=20pt,parsep=10pt
     \item  P \item Q \item X  \item R \item S
  \end{enumext}

\columnbreak

\item Using [columns=2,mini-env] here DOS

  \begin{enumext}[columns=2,mini-env=0.4\linewidth,nosep]%,topsep=10pt,parsep=30pt
    \item  P \item Q \item RS \item T \item U \footnote{ditto}
    \miniright
    \rule{30pt}{20pt}\par
    P Tikz image here
  \end{enumext}

\item Using [columns=1] here below

\begin{enumext}[columns=1,nosep]%,topsep=20pt,parsep=10pt
     \item  P \item Q \item X  \item R \item S \footnote{Otto}
  \end{enumext}

\end{enumext}

After enumext XXXXXXXXXXXXXXXXXXXX


%\end{document}


%\ExplSyntaxOff
\typeout{\baselineskip is \meaning\baselineskip, \the\baselineskip}
\begin{enumext}[list-indent=3cm,listparindent=-1cm,list-offset=3cm,partopsep=0pt,itemsep=0pt,columns=1]%,,itemsep=0pt

\item Vertical spacing should be kept constant on all interior levels (or
appear to be constant). Changing the font size or line spacing should
not affect the output (much).

Vertical spacing should be kept constant on all interior levels (or
appear to be constant). Changing the font size or line spacing should
not affect the output (much).

Vertical spacing should be kept constant on all interior levels (or
appear to be constant). Changing the font size or line spacing should
not affect the output (much).

Vertical spacing should be kept constant on all interior levels (or
appear to be constant). Changing the font size or line spacing should
not affect the output (much).

  \begin{enumext}[columns=1]%,topsep=10pt,parsep=0pt,partopsep=0cm
     \item  P \item Q \item Q  \item R \item S
  \end{enumext}

\item Using [columns=2] here

\begin{enumext}[columns=2]%,topsep=20pt,parsep=10pt
     \item  P \item Q \item X  \item R \item S
  \end{enumext}

\item Using [columns=5] here

\begin{enumext}[columns=5]%,topsep=20pt,parsep=10pt
     \item  P \item Q \item X  \item R \item S
  \end{enumext}

\item Using [columns=2,mini-env] here

  \begin{enumext}[columns=2,mini-env=0.4\linewidth]%,topsep=10pt,parsep=30pt
    \item  P \item Q \item RS \item T \item U
    \miniright
    \rule{30pt}{20pt}\par
    P Tikz image here
  \end{enumext}

\item Using [columns=1,mini-env] here

  \begin{enumext}[columns=1,mini-env=0.4\linewidth]%,topsep=10pt,parsep=30pt
    \item  P \item Q \item RS \item T \item U
    \miniright
    \rule{30pt}{20pt}\par
    P Tikz image here
  \end{enumext}

\end{enumext}


\section{Default values}

Vertical spacing should be kept constant on all interior levels (or
appear to be constant). Changing the font size or line spacing should
not affect the output (much).

Before enumext XXX
%\setenumext[level,2]{nosep}
\begin{enumext}[nosep]%[topsep=0pt,parsep=10pt,partopsep=0pt,itemsep=0pt,columns=1]%,,itemsep=0pt

\item Using [columns=1] here UNO

  \begin{enumext}[columns=1,nosep]%,topsep=10pt,parsep=0pt,partopsep=0cm
     \item  P \item Q \item Q  \item R \item S
  \end{enumext}

\item Using [columns=2] here

\begin{enumext}[columns=2]%,topsep=20pt,parsep=10pt
     \item  P \item Q \item X  \item R \item S
  \end{enumext}

\item Using [columns=5] here

\begin{enumext}[columns=5]%,topsep=20pt,parsep=10pt
     \item  P \item Q \item X  \item R \item S
  \end{enumext}

\item Using [columns=5,nosep] here

\begin{enumext}[columns=5,nosep]%,topsep=20pt,parsep=10pt
     \item  P \item Q \item X  \item R \item S
  \end{enumext}


\item Using [columns=2,mini-env,nosep] here VERTICAL

  \begin{enumext}[columns=2,mini-env=0.4\linewidth,nosep]%,topsep=10pt,parsep=30pt
    \item  P \item Q \item RS \item T \item U
    \miniright
    \rule{30pt}{20pt}\par
    %%%P Tikz image here
  \end{enumext}

\item Using [columns=1,mini-env,nosep] here

  \begin{enumext}[columns=1,mini-env=0.4\linewidth,nosep]%,topsep=10pt,parsep=30pt
    \item  P \item Q \item RS \item T \item U
    \miniright
    \rule{30pt}{20pt}\par
    P Tikz image here
  \end{enumext}

\item Using [columns=1,mini-env] here

  \begin{enumext}[columns=1,mini-env=0.4\linewidth]%,topsep=10pt,parsep=30pt
    \item  P \item Q \item RS \item T \item U
    \miniright
    \rule{30pt}{20pt}\par
    A Tikz image here
  \end{enumext}

\item Using [columns=1] here XXX

  \begin{enumext}[columns=1]%,topsep=10pt,parsep=30pt
    \item  P \item Q \item RS \item T \item U
  \end{enumext}

\item Using [columns=1,mini-env] here

  \begin{enumext}[columns=1,mini-env=0.4\linewidth]%,topsep=10pt,parsep=30pt
    \item  P \item Q \item RS \item T \item U
    \miniright
    \rule{30pt}{20pt}\par
    A Tikz image here
  \end{enumext}

\item Using [columns=1] No ,mini-env here

  \begin{enumext}[columns=1]%,topsep=10pt,parsep=30pt
    \item  P \item Q \item RS \item T \item U
  \end{enumext}

\item Using [columns=1,mini-env] here

  \begin{enumext}[columns=1,mini-env=0.4\linewidth]%,topsep=10pt,parsep=30pt
    \item  P \item Q \item RS \item T \item U
    \miniright
    \rule{30pt}{20pt}\par
    A Tikz image here
  \end{enumext}

\end{enumext}
After enumext XXXXXXXXXXXXXXXXXXXX

\newpage

\section{Default values [columns=2] VERTICAL MODE}

Vertical spacing should be kept constant on all interior levels (or
appear to be constant). Changing the font size or line spacing should
not affect the output (much).

Before enumext AAA%%\par
\begin{enumext}[columns=2,below={8pt}]%[topsep=0pt,parsep=0pt,partopsep=0pt,itemsep=0pt,columns=1]%,,itemsep=0pt

\item Using [columns=2] here UNO

  \begin{enumext}[columns=2]
     \item  P \item Q \item Q  \item R \item S
  \end{enumext}

\item Using [columns=1] here

\begin{enumext}[columns=1]
     \item  P \item Q \item X  \item R \item S
  \end{enumext}

\item Using [columns=5] here SSS

\begin{enumext}[columns=5,columns-sep=3pt]%,nosep,topsep=20pt,parsep=10pt
     \item  P \item Q \item X  \item R \item S
  \end{enumext}

\item Using [columns=2,mini-env] here

  \begin{enumext}[columns=2,mini-env=0.4\linewidth]%,topsep=10pt,parsep=30pt
    \item  P \item Q \item RS \item T \item U
    \miniright
    \rule{30pt}{20pt}\par
    P Tikz image here
  \end{enumext}

\columnbreak

\item Using [columns=2,mini-env] here HHHH

  \begin{enumext}[columns=1,mini-env=0.4\linewidth]%,topsep=10pt,parsep=30pt
     \item  P \item Q \item Q  \item R \item S
    \miniright
    \rule{30pt}{20pt}\par
    P Tikz image here
  \end{enumext}

\item Using [columns=2,mini-env] here

  \begin{enumext}[columns=1,mini-env=0.4\linewidth]%,topsep=10pt,parsep=30pt
    \item  P \item Q \item RS \item T \item U
    \miniright
    \rule{30pt}{20pt}\par
    A Tikz image here
  \end{enumext}

\end{enumext}

After enumext XXXXXXXXXXXXXXXXXXXX

\newpage


\section{Default values [columns=2] HORIZONTAL MODE}

Vertical spacing should be kept constant on all interior levels (or
appear to be constant). Changing the font size or line spacing should
not affect the output (much).

Before enumext AAA%%\par
\begin{enumext}[columns=2,below={8pt}]%[topsep=0pt,parsep=0pt,partopsep=0pt,itemsep=0pt,columns=1]%,,itemsep=0pt

\item Using [columns=2] here UNO
  \begin{enumext}[columns=2]
     \item  P \item Q \item Q  \item R \item S
  \end{enumext}

\item Using [columns=1] here
\begin{enumext}[columns=1]
     \item  P \item Q \item X  \item R \item S
  \end{enumext}

\item Using [columns=5] here SSS
\begin{enumext}[columns=5,columns-sep=3pt]%,nosep,topsep=20pt,parsep=10pt
     \item  P \item Q \item X  \item R \item S
  \end{enumext}

\item Using [columns=2,mini-env] here
  \begin{enumext}[columns=2,mini-env=0.4\linewidth]%,topsep=10pt,parsep=30pt
    \item  P \item Q \item RS \item T \item U
    \miniright
    \rule{30pt}{20pt}\par
    P Tikz image here
  \end{enumext}

\columnbreak

\item Using [columns=2,mini-env] here HHHH
  \begin{enumext}[columns=1,mini-env=0.4\linewidth]%,topsep=10pt,parsep=30pt
     \item  P \item Q \item Q  \item R \item S
    \miniright
    \rule{30pt}{20pt}\par
    P Tikz image here
  \end{enumext}

\item Using [columns=2,mini-env] here
  \begin{enumext}[columns=1,mini-env=0.4\linewidth]%,topsep=10pt,parsep=30pt
    \item  P \item Q \item RS \item T \item U
    \miniright
    \rule{30pt}{20pt}\par
    A Tikz image here
  \end{enumext}

\end{enumext}

After enumext.


\section{KEYANS}

Before
\setenumext[keyans]{show-ans=true}
\begin{enumext}[nosep,itemsep=0pt,align=center,save-ans=mytest,check-ans=true,columns=2,save-ref=true,widest=99,] %show-ans
\item Fourth type of question, a problem with numerical response \anskey{$\sqrt{2}$} % 4

\item* Fourth type of question, a problem with numerical response \anskey{$\sqrt{3}$} % 5

\item* First type of questions % 1
    \begin{keyans}[itemindent=0.5cm,start=5,columns=1,show-length]
        \item value
        \item*[$x=5$]correct
        \item value
        \item[Y]
            MMMMM
        \begin{enumext*}[columns=1]
                    \item[X] A a list here :) ... all work OK %%\anskey{OO}
                 \end{enumext*}
    \end{keyans}
\item* Second type of questions % 2
    \begin{enumext}[no-store,label=\Roman*.] % no-store,
        \item $2\alpha+2\delta=90^{\circ}$
        \item $\alpha=\delta$
        \item $\angle EDF=45^{\circ}$
    \end{enumext}
    \begin{keyans}[columns=2,itemindent=0cm]
            \item I only \par OTRA COSA ES CON
            \item*      II only CORRECT
            \item I and II only
            \item I and III only
            \item I, II, and III
    \end{keyans}

\columnbreak

\item Third type of questions % 3
        \begin{enumext}[no-store,list-offset=3cm,label=(\arabic*)]
            \item $2\alpha+2\delta=90^{\circ}$
            \item $\angle EDF=45^{\circ}$
        \end{enumext}
    \begin{keyans}
        \item Alternativa \verb+A+
        \item Alternativa B
        \item*[XZ] Alternativa C CORRECT
        \item Alternativa D
        \item Alternativa E
    \end{keyans}

\item* Fourth type of question, a problem with numerical response \anskey{$\sqrt{2}$} % 4

\item Fourth type of question, a problem with numerical response \anskey{$\sqrt{3}$} % 5

%\columnbreak

\item Fourth type of question, a problem with numerical response \anskey{$\sqrt{5}$}
\item Question use mini-env key (image on left side).

    \begin{keyans}[mini-env=0.3\linewidth, columns=2]
        \item value
        \item value
        \item value
        \item*[$x=5$] correct
        \item value
        \miniright
        \includegraphics[scale=0.1]{example-image-a}
        %\fbox{A Tikz image here}
    \end{keyans}

\item Question with image on left side \anskey{$\sqrt{56}$}

%\begin{minipage}[t]{0.7\linewidth}
    %\begin{keyans}[columns=1]
          %\item value
          %\item value
          %\item value
          %\item*[$x=5$] correct
          %\item value
    %\end{keyans}
    %\end{minipage}\hfill\begin{minipage}[t]{0.3\linewidth}\vspace{0pt} % anchor for [t]
          %\centering
          %\includegraphics[scale=0.20]{example-image-a}
          %\fbox{Tikz image}
%\end{minipage}

\item Question with image on left side

    \begin{keyans}[noitemsep,columns=1, mini-env=0.3\linewidth]
          \item value
          \item value
          \item value
          \item*[$x=5$] correct
          \item value
          \miniright
          \includegraphics[scale=0.20]{example-image-a}
          Tikz image
          %\fbox{Tikz image}
    \end{keyans}

\item Other Question with image on left side

    \begin{keyans}[noitemsep,columns=1, mini-env=0.3\linewidth]
          \item value
          \item value
          \item value
          \item*[$x=5$] correct
          \item value
          \miniright
          \includegraphics[scale=0.20]{example-image-a}
          %%Tikz image
          %\fbox{Tikz image}
    \end{keyans}
\end{enumext}

\printkeyans[columns=2]{mytest}

%%%\stop

\section{Common layout for images or tables XXX}


\begin{enumext}[save-ans=new,show-ans=true,check-ans=true, save-ref=true,nosep,,ref={[\S\Roman*]},before=\setenumext[keyans]{font=\small,list-offset=0cm,wrap-label={\textcolor{gray}{##1}},nosep}]

\item Third type of questions

    \begin{keyans}
        \item Alternativa \verb+A+
        \item Alternativa B
        \item*[XZ] Alternativa C CORRECT
    \end{keyans}

\item Type of questions (these need manual tuning for alternatives,
      usually carrying pictures or tables) \label{HOLA}

 \begin{keyanspic}[3,2]
    \anspic*{\includegraphics[scale=0.15]{example-image-a}}
   \anspic[note]{\includegraphics[scale=0.15]{example-image-a}}
   \anspic{\includegraphics[scale=0.10]{example-image-a}}
   \anspic{\includegraphics[scale=0.15]{example-image-a}}
   \anspic{\includegraphics[scale=0.15]{example-image-a}}

 \end{keyanspic}

\item Type of questions (these need manual tuning for alternatives,
      usually carrying pictures or tables)

 \begin{keyanspic}[3,2]
   \anspic[note]{\includegraphics[scale=0.15]{example-image-a}}
   \anspic{\includegraphics[scale=0.10]{example-image-a}}
   \anspic{\includegraphics[scale=0.15]{example-image-a}}
   \anspic{\includegraphics[scale=0.15]{example-image-a}}
   \anspic*{\includegraphics[scale=0.15]{example-image-a}}
 \end{keyanspic}

\item Type of questions (these need manual tuning for alternatives,
      usually carrying pictures or tables)

  \begin{keyanspic}[2,3]
    \anspic{\includegraphics[scale=0.15]{example-image-a}}
    \anspic[note]{\includegraphics[scale=0.15]{example-image-a}}
    \anspic{\includegraphics[scale=0.15]{example-image-a}}
    \anspic*[note]{\includegraphics[scale=0.25]{example-image-a}}
    \anspic{\includegraphics[scale=0.15]{example-image-a}}
  \end{keyanspic}

\item Type of questions (these need manual tuning for alternatives,
      usually carrying pictures or tables)

  \begin{keyanspic}
    \anspic{\includegraphics[scale=0.15]{example-image-a}}
    \anspic*{\includegraphics[scale=0.15]{example-image-b}}
    \anspic{\includegraphics[scale=0.15]{example-image-a}}
    \anspic{\includegraphics[scale=0.15]{example-image-a}}
    \anspic[Note]{\includegraphics[scale=0.15]{example-image-a}}
  \end{keyanspic}

\item Third type of questions

    \begin{keyans}[columns=5]
        \item A
        \item B
        \item*[XZ] Correct
        \item D
        \item E
    \end{keyans}


\item Type of questions (these need manual tuning for alternatives,
      usually carrying pictures or tables)

  \begin{keyanspic}
    \anspic{\includegraphics[scale=0.15]{example-image-a}}
    \anspic{\includegraphics[scale=0.15]{example-image-a}}
    \anspic*[Note]{\includegraphics[scale=0.15]{example-image-a}}
    \anspic[Note]{\includegraphics[scale=0.15]{example-image-a}}
  \end{keyanspic}

\item Third type of questions

    \begin{keyans}[columns=5]
        \item P
        \item Q
        \item*[R] S
        \item T
        \item U
    \end{keyans}
\end{enumext}

The ref \ref{HOLA} to \ref{new:6} and the answer is \getkeyans{new:6}\par

\getkeyans{new:1}\par
\getkeyans{new:2}\par
\getkeyans{new:3}\par
\getkeyans{new:4}\par
\getkeyans{new:5}\par
\getkeyans{new:6}\par
\printkeyans[columns=2]{new}

% Temp end doc for test
%\end{document}

\section{List of exercises}

\begin{enumext}[wrap-label={\tikz[scale=0.4]\duck[signpost=\scalebox{0.6}{#1}];},save-ans=Worksheet,check-ans=true,columns=2,nosep,save-ref=true]%
\item Factor $x^{2}-2x+1$  \anskey{$\left(x-1\right)^{2}$}
\item Factor $3x+3y+3z$    \anskey{$3(x+y+z)$}
\item True False
  \begin{enumext}
    \item $\alpha > \delta$ \anskey{False}
    \item \LaTeX2e\ is cool? \anskey{Very True!}
  \end{enumext}
\item Related to Linux
  \begin{enumext}
    \item You use linux? \anskey{Yes}
    \item Usually uses the package manager? \anskey{Yes, dnf}
    \item Rate the following package and class
      \begin{enumext}
        \item xsim-exam \anskey{doesn't exist for now :(}
        \item xsim \anskey{very good}
        \item exsheets  \anskey{obsolete}
      \end{enumext}
  \end{enumext}

\end{enumext}

\subsection{Response to exercises}

The answer of exercises \ref{Worksheet:8} is  \getkeyans{Worksheet:8}

\printkeyans{Worksheet}


\begin{enumext}[wrap-label={\tikz[scale=0.4]\duck[signpost=\scalebox{0.6}{#1}];},save-ans=Worksheet,resume,check-ans=true,columns=2,nosep,save-ref=true]%
\item Factor $x^{2}-2x+1$  \anskey{$\left(x-1\right)^{2}$}
\item Factor $3x+3y+3z$    \anskey{$3(x+y+z)$}
\item True False
  \begin{enumext}
    \item $\alpha > \delta$ \anskey{False}
    \item \LaTeX2e\ is cool? \anskey{Very True!}
  \end{enumext}
\item Related to Linux
  \begin{enumext}
    \item You use linux? \anskey{Yes}
    \item Usually uses the package manager? \anskey{Yes, dnf}
    \item Rate the following package and class
      \begin{enumext}
        \item xsim-exam \anskey{doesn't exist for now :(}
        \item xsim \anskey{very good}
        \item exsheets  \anskey{obsolete}
      \end{enumext}
  \end{enumext}

\end{enumext}

\section{continue}

\begin{enumext}[save-ans=mytest,resume, check-ans=true,save-ref=true]

\item Third type of questions XXXX

    \begin{keyans}
        \item[OOOO] Alternativa \verb+A+
        \item Alternativa B
        \item*[XZ] Alternativa C CORRECT
    \end{keyans}

\item Type of questions (these need manual tuning for alternatives,
      usually carrying pictures or tables)

 \begin{keyanspic}[3,2]
    \anspic*{\includegraphics[scale=0.15]{example-image-a}}
   \anspic[note]{\includegraphics[scale=0.15]{example-image-a}}
   \anspic{\includegraphics[scale=0.10]{example-image-a}}
   \anspic{\includegraphics[scale=0.15]{example-image-a}}
   \anspic{\includegraphics[scale=0.15]{example-image-a}}

 \end{keyanspic}

\item Type of questions (these need manual tuning for alternatives,
      usually carrying pictures or tables)

 \begin{keyanspic}[3,2]
   \anspic[note]{\includegraphics[scale=0.15]{example-image-a}}
   \anspic{\includegraphics[scale=0.10]{example-image-a}}
   \anspic{\includegraphics[scale=0.15]{example-image-a}}
   \anspic{\includegraphics[scale=0.15]{example-image-a}}
   \anspic*{\includegraphics[scale=0.15]{example-image-a}}
 \end{keyanspic}

\item Type of questions (these need manual tuning for alternatives,
      usually carrying pictures or tables)

  \begin{keyanspic}[2,3]
    \anspic{\includegraphics[scale=0.15]{example-image-a}}
    \anspic[note]{\includegraphics[scale=0.15]{example-image-a}}
    \anspic{\includegraphics[scale=0.15]{example-image-a}}
    \anspic*[note]{\includegraphics[scale=0.25]{example-image-a}}
    \anspic{\includegraphics[scale=0.15]{example-image-a}}
  \end{keyanspic}

\item Type of questions (these need manual tuning for alternatives,
      usually carrying pictures or tables)

  \begin{keyanspic}
    \anspic{\includegraphics[scale=0.15]{example-image-a}}
    \anspic*{\includegraphics[scale=0.15]{example-image-b}}
    \anspic{\includegraphics[scale=0.15]{example-image-a}}
    \anspic{\includegraphics[scale=0.15]{example-image-a}}
    \anspic[Note]{\includegraphics[scale=0.15]{example-image-a}}
  \end{keyanspic}

\item Third type of questions

    \begin{keyans}[columns=5]
        \item A
        \item B
        \item*[XZ] C
        \item D
        \item E
    \end{keyans}


\item Type of questions (these need manual tuning for alternatives,
      usually carrying pictures or tables)

  \begin{keyanspic}
    \anspic{\includegraphics[scale=0.15]{example-image-a}}
    \anspic{\includegraphics[scale=0.15]{example-image-a}}
    \anspic*[Note]{\includegraphics[scale=0.15]{example-image-a}}
    \anspic[Note]{\includegraphics[scale=0.15]{example-image-a}}
  \end{keyanspic}

\item Third type of questions

    \begin{keyans}[columns=5]
        \item P
        \item Q
        \item*[R] S
        \item T
        \item U
    \end{keyans}
\end{enumext}

\printkeyans{mytest}
\end{document}
























% \begin{macro}{\_@@_zero_parsep:}
%    The function \cs{@@_zero_parsep:} \enquote{adjusted} the value
%    of \cs{l_@@_minipage_after_skip} \ics{parsep} from previus level to  and
% \iffalse
%% Add |\parsep| from previus level for |multicols| in |enumext|.
% \fi
%    \begin{macrocode}
\cs_new_protected:Nn \@@_zero_parsep:
  {
    \int_case:nn { \l_@@_level_int }
      {
        { 2 }{
                \skip_if_eq:nnF { \l_@@_parsep_i_skip } { \c_zero_skip }
                  {
                    \skip_add:Nn \l_@@_minipage_after_skip { 2.8\box_dp:N \strutbox }
                  }
             }
        { 3 }{
                \skip_if_eq:nnF { \l_@@_parsep_ii_skip } { \c_zero_skip }
                  {
                    \skip_add:Nn \l_@@_minipage_after_skip { 2.8\box_dp:N \strutbox }
                  }
             }
        { 4 }{
                \skip_if_eq:nnF { \l_@@_parsep_iii_skip } { \c_zero_skip }
                  {
                    \skip_add:Nn \l_@@_minipage_after_skip { 2.8\box_dp:N \strutbox }
                  }
             }
      }
  }
%    \end{macrocode}
% \end{macro}



     \skip_if_eq:nnF
          { \skip_use:c { l_@@_itemsep_ \@@_level: _skip } } { \c_zero_skip }
          {
            \skip_add:cn { l_@@_itemsep_ \@@_level: _skip }
              {
                  0.65pt plus 0.0pt minus 0.0pt
              }
          }



        \skip_if_eq:nnT
          { \l_@@_topsep_v_skip } { \c_zero_skip }
          {
            \int_compare:nNnT
            { \int_use:N \l_@@_columns_v_int } = { 1 }
            {
              %%\skip_sub:Nn \l_@@_minipage_after_skip
              %%  {
              %%    \box_dp:N \strutbox plus 2.0pt minus 2.0pt
              %%  }
            }
          }

One of the almost imperceptible details is that the value of itemsep is
affected by placing the environment inside multicols, we will add a
small correction value when itemsep is non-zero.



%    \end{macrocode}
%    One of the almost \emph{imperceptible} details is that the value of
%    \ics{itemsep} is affected by placing the environment inside
%    \myenv{multicols},

In some cases the "\itemsep" values of the nested list are stretched possibly due to the use of "\raggedcolumns" and this affects the bottom space when closing the environm

An almost \emph{imperceptible} detail is that in some cases the
§\itemsep§ values of are \emphñ\enquote{stretched}}, possibly due to
the use of §\raggedcolumns§ and this affects the lower space when
closing the environment, which is \emph{\enquote{smaller}} than
expected.

, but hunting the spaces (§glue§) produced by §\topskip§,
% §\splittopskip§ (even §\prevdepth§) is more complicated than it sounds

% (I tried to find some correct value using §\showoutput§ and §\showboxdepth§ I absolutely failed).

we will add a \emph{small correction} value when
%    \ics{itemsep} is non-zero.
%    \begin{macrocode}
        \int_compare:nNnT { \l_@@_level_int } > { 1 }
          {
            \skip_if_eq:nnF
              { \skip_use:c { l_@@_itemsep_ \@@_level: _skip } } { \c_zero_skip }
              {
                \skip_add:cn { l_@@_itemsep_ \@@_level: _skip }
                  {
                      0.625pt plus 0.0pt minus 0.0pt
                  }
              }
          }


%    Now add the value of \ics{parsep} from previus level to \cs{l_@@_multicols_above_skip}.
%    There is a \emph{small space} (not detectable at first sight) above the
%    \myenv{multicols} environment, so we will add a small correction \mymeta{skip}
%    to the above space according to whether the \ics{parsep} value of the
%    previous level is equal to zero. This is necessary since
%    \ics{parsep} from the previous level affects the \emph{vertical spaces}
%    and this is noticeable when using the \mykey{nosep} or \mykey{noitemsep} keys.
% \iffalse
%% Add |\parsep| from previus level for |multicols| in |enumext|.
% \fi
%    \begin{macrocode}
    \int_case:nn { \l_@@_level_int }
      {
        { 2 }{
               \skip_if_eq:nnTF { \l_@@_parsep_i_skip } { \c_zero_skip }
                 {
                   \skip_add:Nn \l_@@_multicols_above_skip
                     {
                       \l_@@_parsep_i_skip + 0.11pt plus 0.0pt minus 0.0pt
                     }
                 }
                 {
                   \skip_add:Nn \l_@@_multicols_above_skip
                     {
                       \l_@@_parsep_i_skip + 0.16pt plus 0.0pt minus 0.0pt
                     }
                 }
             }
        { 3 }{
               \skip_if_eq:nnTF { \l_@@_parsep_ii_skip } { \c_zero_skip }
                 {
                   \skip_add:Nn \l_@@_multicols_above_skip
                     {
                       \l_@@_parsep_ii_skip + 0.11pt plus 0.0pt minus 0.0pt
                     }
                 }
                 {
                   \skip_add:Nn \l_@@_multicols_above_skip
                     {
                       \l_@@_parsep_ii_skip + 0.16pt plus 0.0pt minus 0.0pt
                     }
                 }
             }
        { 4 }{
               \skip_if_eq:nnTF { \l_@@_parsep_iii_skip } { \c_zero_skip }
                 {
                   \skip_add:Nn \l_@@_multicols_above_skip
                     {
                       \l_@@_parsep_iii_skip + 0.11pt plus 0.0pt minus 0.0pt
                     }
                 }
                 {
                   \skip_add:Nn \l_@@_multicols_above_skip
                     {
                       \l_@@_parsep_iii_skip + 0.16pt plus 0.0pt minus 0.0pt
                     }
                 }
             }
      }
  }
%    \end{macrocode}
% \end{macro}

compatible with the \mypkg{hyperref}\cite{hyperref} package

%
% \smallskip
%
% If the \mypkg{hyperref}\cite{hyperref} package is detected,
% \ics{hyperlink} and \ics{hypertarget} will be used, otherwise the
% usual system provided by \hologo{LaTeX} will be used.


% When nesting a \myenv{list} environment within the \myenv{multicols}
% environment the \emph{vertical spaces} above and below the lists are
% lost. Graphically it can be seen as in figure \ref{fig:sixb}.
%
% \begin{figure}[ht]
%   \centering
%   \begin{tikzpicture}[line cap=round,line join=round,x=0.8cm,y=0.8cm,every node/.style={font=\footnotesize}]
%     \draw[color=white] (7.0,6.10) rectangle (17.5,6.5) node[pos=.5,text=black]%
%       { \textcolor{blue}{\texttt{\textbackslash{}multicolsep [±\textbackslash{}topskip]}} };% space above
%     \draw[color=white] (7.0,4.85) rectangle (17.5,5.25) node[pos=.5,text=black]%
%       { \textcolor{blue}{\texttt{\textbackslash{}multicolsep [±\textbackslash{}prevdepth?]}}}; % space below
%     \draw[->,color=lightgray,shorten >=3pt] (5.0,6.3) --node[above,text=optcolor]{\texttt{topsep}} (7.0,6.3);   % glue space = topsep (above)
%     \draw[->,color=lightgray,shorten >=3pt] (5.0,5.05) --node[above,text=optcolor]{\texttt{topsep}} (7.0,5.05); % glue space = topsep (below)
%     \draw[color=lightgray,fill=lightgray,fill opacity=0.25,text=blue,font=\ttfamily] (7.0,6.5) rectangle (17.5,7.00) node[pos=.5,text=blue,font=\small\ttfamily]{Text above list};
%     \draw[color=lightgray,fill=lightgray,fill opacity=0.25] (7.0,5.25) rectangle (11.5,6.10) node[pos=.5,text=black]%
%        {\begin{tabular}{c}
%          \texttt{column one} \\
%          \textcolor{blue}{\emph{nested list}}
%        \end{tabular}};
%     \draw[color=lightgray,fill=lightgray,fill opacity=0.25] (13.5,5.25) rectangle (17.5,6.10) node[pos=.5,text=black]%
%        {\begin{tabular}{c}
%          \texttt{column two} \\
%          \textcolor{blue}{\emph{continued nested list}}
%        \end{tabular}};
%     \draw[color=lightgray,fill=lightgray,fill opacity=0.25,text=blue,font=\ttfamily] (7.0,4.35) rectangle (17.5,4.85) node[pos=0.5,text=blue,font=\small\ttfamily] {Text below list};
%   \end{tikzpicture}
%   \caption{Representation of the \emph{vertical space} in above and below \myenv{multicols} with a nested list.}
%   \label{fig:sixb}
% \end{figure}

%
%
% \section{The way of non-enumerated lists}
%
% It is possible to use (or abuse) the \myenv*{enumext} environment to
% mimic non-enumerated environments such as \myenv{itemize} and
% \myenv{description}, clearly the keys to \emph{\enquote{store answers}}
% and the \myenv*{keyans} and \myenv*{keyanspic} environments lose their
% sense and it is not the focus of the main of this package, but, why not
% to do it?.

% Here I leave as an example other uses of the \myenv*{enumext}
% environment that can be helpful for specific purposes, the source code
% of these examples is at the bottom of this page.


\tl_new:N \g_shenum_fnt_tl
\cs_new:Npn \__shenum_ftntext:n #1
  {
    \exp_args:Nc \exp_not:V \g_shenum_fnt_tl
      {
        \exp_not:N \footnotetext  [\int_use:c { c@\@mpfn }]
      }
      {#1}
  }
\cs_new:Npn \__shenum_xftntext:nn [#1] #2
   {
     \exp_not:V \g_shenum_fnt_tl {  \footnotetext [#1] {#2}  }
   }

\tl_gset:Ne \g_shenum_fnt_tl { \exp_not:V \g_shenum_fnt_tl }


\cs_set_eq:NN \__shenum_ftntext:n \@footnotetext
\cs_set_eq:NN \__shenum_xftntext:nn \@xfootnotenext






\c@footnote

%% LaTeX 2e style
% Code Part I
\ifx\TX@ftn\undefined
   \newtoks\TX@ftn
   \long\def\TX@ftntext#1{%
     \edef\@tempa{\the\TX@ftn\noexpand\footnotetext
                       [\the\csname c@\@mpfn\endcsname]}%
     \global\TX@ftn\expandafter{\@tempa{#1}}}%
   \long\def\TX@xftntext[#1]#2{%
     \global\TX@ftn\expandafter{\the\TX@ftn\footnotetext[#1]{#2}}}
\fi
% Code Part II
\global\TX@ftn\expandafter{\expandafter}\the\TX@ftn
% Code Part III
\let\@footnotetext\TX@ftntext\let\@xfootnotenext\TX@xftntext
%% LaTeX3 style
% Code Part I
\tl_new:N \g_shenum_fnt_tl
\cs_new_protected:Npn \__shenum_ftntext:n #1
  {
    \tl_gset:Ne \g_shenum_fnt_tl
      {
        \exp_not:N \footnotetext [ \int_use:c { c@footnote } ] {#1}
      }
  }
\cs_new_protected:Npn \__shenum_xftntext:nn [#1] #2
  {
    \tl_gset:Ne \g_shenum_fnt_tl
      {
        \exp_not:N \footnotetext [#1] {#2}
      }
  }
% Code Part II
\tl_if_empty:NF \g_shenum_fnt_tl
  {
    \g_shenum_fnt_tl
    \tl_gclear:N \g_shenum_fnt_tl
  }
% Code Part III
\cs_set_eq:cN { @footnotetext } \__shenum_ftntext:n
\cs_set_eq:cN { @xfootnotenext } \__shenum_xftntext:nn





newlength, settowidth
desitemcwd, textbf



           \int_compare:nNnT { \lastnodetype } = { 11 }
              {
                %\int_show:N \lastnodetype
                %\skip_show:N \lastskip
                %%\showthe\lastskip
                %\vskip-\lastskip%%\unskip%%%\unskip\unskip
%                       %\skip0\lastskip
                %\unskip
              }


It is possible to use (or abuse) the "enumext" environment to mimic
non-enumerated environments such as "itemize" and "description", of
course here many of the keys and the "keyans" and "keyanspic"
environment become meaningless.
