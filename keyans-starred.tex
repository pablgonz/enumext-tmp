% \subsection{The \env{keyans*} environment}\label{env:keyans-starred}
%
% \subsubsection{Functions for item box width}
%
% \begin{macro}{\@@_starred_columns_set_viii:}
%    We set the default value for the width of the box containing the
%    content of the items and create \ics*{itemwidth} in a public form.
% \iffalse
%% Define default columns width for |keyans*|
% \fi
%    \begin{macrocode}
\cs_new_protected:Nn \@@_starred_columns_set_viii:
  {
    \dim_compare:nNnT { \l_@@_columns_sep_viii_dim } = { \c_zero_dim }
      {
        \dim_set:Nn \l_@@_columns_sep_viii_dim
          {
            ( \l_@@_labelwidth_viii_dim + \l_@@_labelsep_viii_dim )
            / \l_@@_columns_viii_int
          }
      }
    \int_set:Nn \l_@@_tmpa_viii_int { \l_@@_columns_viii_int - \c_one_int }
    \dim_set:Nn \l_@@_item_width_viii_dim
      {
        ( \linewidth - \l_@@_columns_sep_viii_dim * \l_@@_tmpa_viii_int )
        / \l_@@_columns_viii_int - \l_@@_labelwidth_viii_dim
        - \l_@@_labelsep_viii_dim
      }
    \dim_zero_new:N \itemwidth
  }
%    \end{macrocode}
% \end{macro}
%
% \begin{macro}{\@@_starred_joined_item_viii:n}
%    The function \cs{@@_starred_joined_item_viii:n} will set the \emph{width} of the
%    box in which the content passed to \ics*{item}\myparg{number} will be
%    stored together with the value of \ics*{itemwidth}.
%
% \iffalse
%% Set width for joined item's for |keyans*|
% \fi
%    \begin{macrocode}
\cs_new_protected:Npn \@@_starred_joined_item_viii:n #1
  {
    \int_set:Nn \l_@@_joined_item_viii_int {#1}
    \int_compare:nNnT { \l_@@_joined_item_viii_int } > { \l_@@_columns_viii_int }
      {
        \msg_warning:nnee { enumext } { item-joined }
          { \int_use:N \l_@@_joined_item_viii_int }
          { \int_use:N \l_@@_columns_viii_int }
        \int_set:Nn \l_@@_joined_item_viii_int
          {
            \l_@@_columns_viii_int - \l_@@_item_column_pos_viii_int + \c_one_int
          }
      }
    \int_compare:nNnT
      { \l_@@_joined_item_viii_int }
        >
      { \l_@@_columns_viii_int - \l_@@_item_column_pos_viii_int + \c_one_int }
      {
        \msg_warning:nnee { enumext } { item-joined-columns }
          { \int_use:N \l_@@_joined_item_viii_int }
          {
            \int_eval:n
              { \l_@@_columns_viii_int - \l_@@_item_column_pos_viii_int + \c_one_int }
          }
        \int_set:Nn \l_@@_joined_item_viii_int
          {
            \l_@@_columns_viii_int - \l_@@_item_column_pos_viii_int + \c_one_int
          }
      }
%    \end{macrocode}
%    Only need if |#1 > 1| (default are set before).
%    \begin{macrocode}
    \int_compare:nNnTF { \l_@@_joined_item_viii_int } > { \c_one_int }
      {
        \int_set_eq:NN \l_@@_joined_item_aux_viii_int \l_@@_joined_item_viii_int
        \int_decr:N \l_@@_joined_item_aux_viii_int
        \int_add:Nn \l_@@_item_column_pos_viii_int { \l_@@_joined_item_aux_viii_int }
        \int_gadd:Nn \g_@@_item_count_all_viii_int { \l_@@_joined_item_aux_viii_int }
        \dim_set:Nn \l_@@_joined_width_viii_dim
          {
            \l_@@_item_width_viii_dim * \l_@@_joined_item_viii_int
            + (  \l_@@_labelwidth_viii_dim + \l_@@_labelsep_viii_dim
               + \l_@@_columns_sep_viii_dim
              )*\l_@@_joined_item_aux_viii_int
          }
        \dim_set_eq:NN \itemwidth \l_@@_joined_width_viii_dim
      }
      {
        \dim_set_eq:NN \l_@@_joined_width_viii_dim \l_@@_item_width_viii_dim
        \dim_set_eq:NN \itemwidth \l_@@_item_width_viii_dim
      }
  }
%    \end{macrocode}
% \end{macro}
%
% \begin{macro}{\@@_start_mini_viii:}
%
% The implementation of the \mykey{mini-env} key support is almost
% identical to the one used in the \myenv*{enumext} and \myenv*{keyans}
% environments, the difference is that the \myvarenv*{__@@_mini_env*}
% environment on the \emph{\enquote{right side}} is executed
% \emph{\enquote{after}} closing the environment, so it is necessary to
% make a global copy of the variable \cs{l_@@_minipage_right_viii_dim} in
% the variable \cs{g_@@_minipage_right_viii_dim}.
%
% \iffalse
%% Start support for |mini-env| key in |keyans*| environment
% \fi
%    \begin{macrocode}
\cs_new_protected:Nn \@@_start_mini_viii:
  {
    \dim_compare:nNnT { \l_@@_minipage_right_viii_dim } > { \c_zero_dim }
      {
        \dim_set:Nn \l_@@_minipage_left_viii_dim
          {
            \linewidth
            - \l_@@_minipage_right_viii_dim
            - \l_@@_minipage_hsep_viii_dim
          }
        \bool_set_true:N \l_@@_minipage_active_viii_bool
        \dim_gset_eq:NN
          \g_@@_minipage_right_viii_dim
          \l_@@_minipage_right_viii_dim
        \@@_mini_addvspace_viii:
        \nointerlineskip\noindent
        \begin{@@_mini_env*}{ \l_@@_minipage_left_viii_dim }
      }
   }
%    \end{macrocode}
% \end{macro}
%
% \begin{macro}{\@@_stop_mini_viii:}
%   The function \cs{@@_stop_mini_viii:} closes the
%   \myvarenv*{__@@_mini_env*} environment on the left side, applies
%   |\hfill| and sets the value of the variable
%   \cs{g_@@_minipage_active_viii_bool} to true which will be used in the
%   function \cs{@@_after_star_env:nn} to execute the
%   \myvarenv*{__@@_mini_env*} on the \emph{\enquote{right side}}.
% \iffalse
%% Stop support for |mini-env| key in |keyans*| environment
% \fi
%    \begin{macrocode}
\cs_new_protected:Nn \@@_stop_mini_viii:
  {
    \bool_if:NT \l_@@_minipage_active_viii_bool
      {
        \end{@@_mini_env*}
        \hfill
        \bool_gset_true:N \g_@@_minipage_active_viii_bool
      }
   }
%    \end{macrocode}
%
%    Finally we execute code passed to the \mykey{miniright} key
%    stored in the variable \cs{g_@@_miniright_code_viii_tl} in the
%    \myvarenv*{__@@_mini_env*} environment on the \emph{\enquote{right side}}.
% \iffalse
%% Exec |miniright| key in |keyans*| environment
% \fi
%    \begin{macrocode}
\@@_after_env:nn {keyans*}
  {
    \bool_if:NT \g_@@_minipage_active_viii_bool
      {
        \begin{@@_mini_env*}{ \g_@@_minipage_right_viii_dim }
          \par\addvspace { \g_@@_minipage_right_skip }
          \bool_if:NF \g_@@_minipage_center_viii_bool
            {
              \centering
            }
          \tl_use:N \g_@@_miniright_code_viii_tl % the code
        \end{@@_mini_env*}
        \par\addvspace{ \g_@@_minipage_after_skip }
      }
    \bool_gset_false:N \g_@@_minipage_active_viii_bool
    \bool_gset_true:N \g_@@_minipage_center_viii_bool
    \tl_gclear:N \g_@@_miniright_code_viii_tl
    \dim_gzero:N \g_@@_minipage_right_viii_dim
  }
%    \end{macrocode}
% \end{macro}
%
% \begin{macro}{keyans*}
%    First we will generate the environment and we will give a
%    temporary definition to \cs{@@_stop_item_tmp_viii:} equal to
%    \ics{noindent} and next to \ics{item} equal to
%    \cs{@@_start_item_tmp_viii:} which we will redefine later.
% \iffalse
%% Define |keyans*|
% \fi
%    \begin{macrocode}
\NewDocumentEnvironment{keyans*}{ o }
  {
    \@@_safe_exec_viii:
    \@@_parse_keys_viii:n {#1}
    \@@_before_list_viii:
    \@@_start_store_level_viii:
    \@@_start_list:nn { }
      {
        \@@_list_arg_two_viii:
        \@@_before_keys_exec_viii:
      }
      \@@_starred_columns_set_viii:
      \item[] \scan_stop:
      \cs_set_eq:NN \@@_stop_item_tmp_viii: \noindent
      \cs_set_eq:NN \item \@@_start_item_tmp_viii:
  }
  {
    \@@_stop_item_tmp_viii:
    \@@_remove_extra_parsep_viii:
    \@@_stop_list:
    \@@_stop_store_level_viii:
    \@@_after_list_viii:
  }
%    \end{macrocode}
% \end{macro}
%
% \begin{macro}{\@@_safe_exec_viii:}
%    First check the maximum nesting level for the \myenv*{keyans*}
%    environment.
% \iffalse
%% Test deep-level for |keyans*| and set bool vars.
% \fi
%    \begin{macrocode}
\cs_new_protected:Nn \@@_safe_exec_viii:
  {
    \int_incr:N \l_@@_keyans_level_h_int
    \int_compare:nNnT { \l_@@_keyans_level_h_int } > { 1 }
      {
        \msg_error:nn { enumext } { nested }
      }
    % Set false for interfering with enumext nested in keyans* (yes, its possible and crayze)
    \bool_set_false:N \l_@@_store_active_bool
    \int_compare:nNnT { \l_@@_level_int } > { 1 }
      {
        \msg_error:nn { enumext } { keyans-wrong-level }
      }
  }
%    \end{macrocode}
% \end{macro}
%
% \begin{macro}{\@@_parse_keys_viii:n}
%    Parse \myoarg[type=tt]{key \textnormal{\textcolor{gray}{=}} val}
%    for \myenv*{keyans*}.
% \iffalse
%% Parse |key=val| and store keys for |keyans*|.
% \fi
%    \begin{macrocode}
\cs_new_protected:Npn \@@_parse_keys_viii:n #1
  {
    \tl_if_novalue:nF {#1}
      {
        \keys_set:nn { enumext / keyans* } {#1}
      }
  }
%    \end{macrocode}
% \end{macro}
%
% \begin{macro}{\@@_before_list_viii:}
%    The function \cs{@@_before_list_viii:} will add the vertical
%    spacing on the environment if the \mykey{above} key is active next to
%    the \mymarg[type=tt]{code} defined by the \mykey{before*} key if
%    it is active, the call the function \cs{@@_start_mini_viii:} handle by
%    \mykey{mini-env}.
%    \begin{macrocode}
\cs_new_protected:Nn \@@_before_list_viii:
  {
    \@@_vspace_above_viii:
    \@@_before_args_exec_viii:
    \@@_start_mini_viii:
  }
%    \end{macrocode}
% \end{macro}
%
% \begin{macro}{\@@_after_list_viii:}
%    The function \cs{@@_after_list:} first call the function \cs{@@_stop_mini_viii:},
%    then apply the \mymarg[type=tt]{code} handled by the \mykey{after} key
%    together with the \emph{vertical space} handled by the \mykey{below} key if
%    they are present.
%
%    \begin{macrocode}
\cs_new_protected:Nn \@@_after_list_viii:
  {

    \@@_stop_mini_viii:
    \@@_after_stop_list_viii:
    \@@_vspace_below_viii:
  }
%    \end{macrocode}
% \end{macro}
%
% \subsubsection{The command \cs{item} in \env{keyans*}}\label{cmd:item-boxed-viii}
%
% \begin{macro}{\@@_start_item_tmp_viii:}
%
% First we will call the function \cs{@@_stop_item_tmp_viii:} that we
% will redefine later, we will increment the value of
% \cs{l_@@_item_column_pos_viii_int} that will count the item's by rows and
% the value of \cs{g_@@_item_count_all_viii_int} that will count the
% total of item's in the environment. After that we will call the
% function \cs{@@_item_peek_args_viii:} that will handle the
% arguments passed to \ics*{item}.
% \iffalse
%% % Start boxed item's for |keyans*|
% \fi
%    \begin{macrocode}
\cs_new_protected_nopar:Nn \@@_start_item_tmp_viii:
  {
    \@@_stop_item_tmp_viii:
    \int_incr:N \l_@@_item_column_pos_viii_int
    \int_gincr:N \g_@@_item_count_all_viii_int
    \@@_item_peek_args_viii:
  }
%    \end{macrocode}
% \end{macro}
%
% \begin{macro}{\@@_item_peek_args_viii:}
%
%    The function \cs{@@_item_peek_args_viii:} will handle the \ics*{item}\myparg{number}.
%    Look for the argument \enquote{\textcolor{MediumOrchid}{\ttfamily (}}, if it is
%    present we will call the function \cs{@@_joined_item_viii:w} \myparg{number}, which is in charge of joining the item's
%    in the same row, in case they are not present we will set the default
%    value \textcolor{gray}{\ttfamily(\mydim{1})}.
% \iffalse
%% Peek argument for \item()*[][] for |keyans*|
% \fi
%    \begin{macrocode}
\cs_new_protected:Nn \@@_item_peek_args_viii:
  {
    \peek_meaning:NTF (
      { \@@_joined_item_viii:w }
      { \@@_joined_item_viii:w (1) }
  }
%    \end{macrocode}
% \end{macro}
%
% \begin{macro}{\@@_joined_item_viii:w}
%
%    The function \cs{@@_joined_item_viii:w} will first call the
%    function \cs{@@_starred_joined_item_viii:n} in charge of setting the
%    \emph{width} of the box that will store the content passed to \ics*{item}. Then we
%    will look for the argument \enquote{\textcolor{MediumOrchid}{\ttfamily *}}, if it is present we will call the
%    function \cs{@@_starred_item_viii:w} otherwise we will call the
%    function \cs{@@_standard_item_viii:w}.
% \iffalse
%% Joined item's for |keyans*|
% \fi
%    \begin{macrocode}
\cs_new_protected:Npn \@@_joined_item_viii:w (#1)
  {
    \@@_starred_joined_item_viii:n {#1}
    \peek_meaning_remove:NTF *
      { \@@_starred_item_viii:w  }
      { \@@_standard_item_viii:w }
  }
%    \end{macrocode}
% \end{macro}
%
% \begin{macro}{\@@_standard_item_viii:w}
%
% The function \cs{@@_standard_item_viii:w} will first look for the
% argument “|[|”, if present it will set the state of the variable
% \cs{l_@@_wrap_label_opt_viii_bool} equal to the state of the variable
% \cs{l_@@_wrap_label_opt_viii_bool} handled by the key
% \mykey{wrap-label*} and finally execute the \emph{non-enumerated}
% version \ics*{item}\myoarg[type=tt]{custom} by means of the function
% \cs{@@_start_item_viii:w}, otherwise we will set the value of the
% variable \cs{l_@@_wrap_label_viii_bool} handled by the
% \mykey{wrap-label} key to true and set the switch |\if@noitemarg| to
% true to execute the enumerated version of \ics*{item} by means of the
% function \cs{@@_start_item_viii:w} |[| \myvarenv{l__@@_label_viii_tl} |]|.
%
% \iffalse
%% Set \item[opt] for |keyans*|
% \fi
%    \begin{macrocode}
\cs_new_protected:Npn \@@_standard_item_viii:w
  {
    \bool_set_false:N \l_@@_item_starred_viii_bool
      \peek_meaning:NTF [
        {
          \bool_set_eq:NN
            \l_@@_wrap_label_viii_bool
            \l_@@_wrap_label_opt_viii_bool
          \@@_start_item_viii:w
        }
        {
          \bool_set_true:N \l_@@_wrap_label_viii_bool
          \legacy_if_set_true:n { @noitemarg }
          \@@_start_item_viii:w [ \l_@@_label_viii_tl ]
        }
  }
%    \end{macrocode}
% \end{macro}
%
% \begin{macro}{\@@_starred_item_viii:w, \@@_starred_item_viii_aux_i:w,
%               \@@_starred_item_viii_aux_ii:w, \@@_starred_item_viii_aux_iii:w,}
%
% The function \cs{@@_starred_item_viii:w} together with the specified
% auxiliary functions |aux_i:w|, |aux_ii:w|, and |aux_iii:w| execute \ics*{item*},
% \ics*{item*}\myoarg[type=tt]{symbol} and
% \ics*{item*}\myoarg[type=tt]{symbol}\myoarg[type=tt]{offset}.
%
% \iffalse
%% Set \item*[sym][sep] for |keyans*|
% \fi
%    \begin{macrocode}
\cs_new_protected:Npn \@@_starred_item_viii:w
  {
    \bool_set_true:N \l_@@_item_starred_viii_bool
    \bool_set_true:N \l_@@_wrap_label_viii_bool
    \peek_meaning:NTF [
      { \@@_starred_item_viii_aux_i:w }
      { \@@_starred_item_viii_aux_ii:w }
  }
\cs_new_protected:Npn \@@_starred_item_viii_aux_i:w [#1]
  {
    \tl_gset:Nn \g_@@_item_symbol_aux_viii_tl {#1}
    \@@_starred_item_viii_aux_ii:w
  }
\cs_new_protected:Npn \@@_starred_item_viii_aux_ii:w
  {
    \peek_meaning:NTF [
      { \@@_starred_item_viii_aux_iii:w }
      {
        \dim_set_eq:NN
          \l_@@_item_symbol_sep_viii_dim
          \l_@@_labelsep_viii_dim
        \legacy_if_set_true:n { @noitemarg }
        \@@_start_item_viii:w [ \l_@@_label_viii_tl ]
      }
  }
\cs_new_protected:Npn \@@_starred_item_viii_aux_iii:w [#1]
  {
    \dim_set:Nn \l_@@_item_symbol_sep_viii_dim {#1}
    \legacy_if_set_true:n { @noitemarg }
    \@@_start_item_viii:w [ \l_@@_label_viii_tl ]
  }
%    \end{macrocode}
% \end{macro}
%
%  \subsubsection*{Real definition of \cs{item}}
%
%  The functions \cs{@@_start_item_viii:w} and \cs{@@_stop_item_viii:} executing the true definition of
%  \ics*{item} inside the \myenv*{keyans*} environment.
%
% \begin{macro}{\@@_start_item_viii:w}
%
%    The first thing we will do is set the value of \cs{@@_stop_item_tmp_viii:} equal to the
%    value of \cs{@@_stop_item_viii:} which we will define later and add the
%    \mypkg{hyperref} compatible \icounter{enumXviii} counter, after that we will
%    start capturing the item content in a box. Here need setting the |\if@hyper@item| switch to
%    \emph{\enquote{true}} for \mypkg{hyperref} compatible. The
%    explanation for this is given by the master Heiko Oberdiek on
%    \href{https://tex.stackexchange.com/a/404911}{\textbackslash refstepcounter\{enumi\} twice (or more) creates destination with the same identifier}.
%
% \iffalse
%% Real |\item| definition for |keyans*| \legacy_if_set_true:n { @hyper@item } :D. (doc p170)
% \fi
%    \begin{macrocode}
\cs_new_protected_nopar:Npn \@@_start_item_viii:w [#1]
  {
    \cs_set_eq:NN \@@_stop_item_tmp_viii: \@@_stop_item_viii:
    \legacy_if:nT { @noitemarg }
      {
        \legacy_if_set_false:n { @noitemarg }
        \legacy_if:nT { @nmbrlist }
          {
            \bool_if:NT \l_@@_hyperref_bool
              {
                \legacy_if_set_true:n { @hyper@item }
              }
            \refstepcounter{enumXviii}
            % code for check-ans
            \bool_if:NT \l_@@_check_ans_bool
              {
                % If true |no-store| key => nested in |enumext|
                \bool_if:NTF \l_@@_store_ans_bool
                  {
                    \int_gadd:cn { g_@@_count_item_ \@@_level: _int }
                      { \int_use:c { g_@@_count_level_ \@@_level: _int } + 1 }
                  }
                  {
                    \int_gincr:N \g_@@_count_item_all_int
                    \int_gincr:N \g_@@_count_level_viii_int
                  }
              }
          }
      }
%    \end{macrocode}
%
%    Here we start capturing |\item| and its contents into a group using the
%    plain form of the \myenv{lrbox} environment. If the state of the
%    variable \cs{l_@@_footnotes_key_bool} is false, we will redefine the
%    command |\footnote|, followed by printing the \mymeta{symbol} defined
%    for |\item*| if it is present and open a new group inside which we
%    execute \mykey{font} key next to |\item| and the keys
%    \mykey{wrap-label}, \mykey{wrap-label*}, \mykey{align}, close the group and execute
%    the key \mykey{labelsep} and then the key \mykey{first}. Finally we
%    open the \myenv{minipage} environment and execute the
%    \mykey{listparindent} key which will be equal to |\parindent|, the
%    \mykey{parsep} key which will be equal to |\parskip| and the
%    \mykey{itemindent} key.
%    \begin{macrocode}
    \group_begin:
      \lrbox{ \l_@@_item_text_viii_box }
        \bool_if:NF \l_@@_footnotes_key_bool
          {
            \@@_renew_footnote:
          }
        \bool_if:NT \l_@@_item_starred_viii_bool
          {
            \tl_if_blank:VT \g_@@_item_symbol_aux_viii_tl
              {
                \tl_gset_eq:NN
                  \g_@@_item_symbol_aux_viii_tl \l_@@_item_symbol_viii_tl
              }
            \mode_leave_vertical:
            \skip_horizontal:n { -\l_@@_item_symbol_sep_viii_dim }
            \makebox[ 0pt ][ r ]{ \g_@@_item_symbol_aux_viii_tl }
            \skip_horizontal:N \l_@@_item_symbol_sep_viii_dim
            \tl_gclear:N \g_@@_item_symbol_aux_viii_tl
          }
        \group_begin:
          \tl_use:N \l_@@_label_font_style_viii_tl
          \bool_if:NTF \l_@@_wrap_label_viii_bool
            {
              \makebox[ \l_@@_labelwidth_viii_dim ][ \l_@@_align_label_viii_str ]
                { \@@_wrapper_label_viii:n {#1} }
            }
            {
              \makebox[ \l_@@_labelwidth_viii_dim ][ \l_@@_align_label_viii_str ]{ #1 }
            }
        \group_end:
        \skip_horizontal:N \l_@@_labelsep_viii_dim
        \tl_use:N \l_@@_after_list_args_viii_tl
        \@@_minipage:w [ t ]{ \l_@@_joined_width_viii_dim  }
          \skip_set_eq:NN \parindent \l_@@_listparindent_viii_dim
          \skip_set_eq:NN \parskip \l_@@_parsep_viii_skip
          \tl_use:N \l_@@_fake_item_indent_viii_tl
   }
%    \end{macrocode}
% \end{macro}
%
% \begin{macro}{\@@_stop_item_viii:}
%   The function \cs{@@_stop_item_viii:} shall terminate with the capture of
%   |\item| and its \mymeta{contents}. Close the environments
%   \myenv{minipage}, \myenv{lrbox} and the group. Then we
%   only have to set the width of the box and print it next to
%   |\footnote|, and add the horizontal and vertical separation between
%   the boxes.
%    \begin{macrocode}
\cs_new_protected_nopar:Nn \@@_stop_item_viii:
  {
        \@@_endminipage:
      \endlrbox
    \group_end:
    \box_set_wd:Nn \l_@@_item_text_viii_box
      {
        \l_@@_joined_width_viii_dim
        + \l_@@_labelwidth_viii_dim
        + \l_@@_labelsep_viii_dim
      }
    \int_set:Nn \hbadness { 10000 }
    \box_use:N \l_@@_item_text_viii_box
    \bool_if:NF \l_@@_footnotes_key_bool
      {
        \@@_print_footnote:
      }
    \int_compare:nNnTF { \l_@@_item_column_pos_viii_int } = { \l_@@_columns_viii_int }
      {
        \par\noindent
        \int_zero:N \l_@@_item_column_pos_viii_int
      }
      { \hspace{ \l_@@_columns_sep_viii_dim } }
  }
%    \end{macrocode}
% \end{macro}
%
% \begin{macro}{\@@_remove_extra_parsep_viii:}
%   Finally we will remove the vertical space equal to |\parsep| when the
%   total number of items is divisible by the number of items in the last
%   row of the environment.
%
% \iffalse
%% Remove extra \parsep in |keyans*|
% \fi
%    \begin{macrocode}
\cs_new_protected:Nn \@@_remove_extra_parsep_viii:
  {
    \int_compare:nNnT
      {
        \int_mod:nn {  \g_@@_item_count_all_viii_int } { \l_@@_columns_viii_int }
      }
      =
      { \c_zero_int }
      {
        \par
        \vspace{ -\l_@@_itemsep_viii_skip }
        \int_gzero:N \g_@@_item_count_all_viii_int
      }
  }
%    \end{macrocode}
% \end{macro}
%
% As we don't want our check to be executed \mykey{check-ans} by levels but on the
% complete list, we will take it out of the \myenv*{keyans*}
% environment using the \emph{\enquote{hook}} function \cs{@@_after_env:nn}.
% \iffalse
%% Execute |check-ans| key out of |keyans*|.
% \fi
%    \begin{macrocode}
\@@_after_env:nn {keyans*}
  {
    \bool_if:NT \g_@@_check_ans_show_h_bool
      {
        \int_compare:nNnT { \l_@@_level_int } = { 0 }
          {
            \@@_check_ans_active_viii:
          }
      }
    \bool_gset_false:N \g_@@_check_ans_show_h_bool
    \tl_gclear:N \g_@@_store_name_tl
  }
%    \end{macrocode}
