% arara: pdflatex
% arara: pdflatex
% arara: clean: { extensions: [ aux, log, out] }
\documentclass{article}
\usepackage[T1]{fontenc}
\usepackage{lmodern}
\usepackage[italian]{babel}
\usepackage{siunitx,amssymb,tikz}
\usetikzlibrary{shapes.geometric}
\sisetup{output-decimal-marker={,}}
\DeclareSIUnit{\angstrom}{\textup{\AA}}
\usepackage[colorlinks]{hyperref}
\usepackage{enumext}
\newcommand*{\mySquared}[1]{%
  \begin{tikzpicture}[baseline=(number.base),square/.style={regular polygon,regular polygon sides=4}]
    \node[square, rounded corners=1pt, inner sep=1pt, draw=none, fill=gray!40] (number) {#1};
  \end{tikzpicture}%
}
\pagestyle{empty}
\begin{document}
\setenumext[keyans]{label=\Alph*,font=\small,nosep,wrap-label={\mySquared{##1}}}
\begin{enumext}[columns=1,save-ans=sabastiano2,mark-pos=left,mark-ans=$\checkmark$,store-ref,show-ans,itemsep=0pt]
 \item La velocità di \SI{1,00e2}{m/s} espressa in \si{km/h} è:
   \begin{keyans}
     \item \SI{36}{km/h}.
     \item* \SI{360}{km/h}.
     \item \SI{27,8}{km/h}.
     \item \SI{3,60e8}{km/h}.
   \end{keyans}

 \item In fisica nucleare si usa l'angstrom (simbolo:
 $\SI{1}{\angstrom} = \SI{1e-10}{m}$) e il fermi o femtometro
 ($\SI{1}{fm} = \SI{1e-15}{m}$). Qual è la relazione tra queste due
 unità di misura?
   \begin{keyans}
     \item* $\SI{1}{\angstrom}=\SI{1e5}{fm}$.
     \item $\SI{1}{\angstrom}=\SI{1e-5}{fm}$.
     \item $\SI{1}{\angstrom}=\SI{1e-15}{fm}$.
     \item $\SI{1}{\angstrom}=\SI{1e3}{fm}$.
   \end{keyans}

 \item La velocità di \SI{1,00e2}{m/s} espressa in \si{km/h} è:
   \begin{keyans}
     \item \SI{36}{km/h}.
     \item* \SI{360}{km/h}.
     \item \SI{27,8}{km/h}.
     \item \SI{3,60e8}{km/h}.
   \end{keyans}

 \item In fisica nucleare si usa l'angstrom (simbolo:
 $\SI{1}{\angstrom} = \SI{1e-10}{m}$) e il fermi o femtometro
 ($\SI{1}{fm} = \SI{1e-15}{m}$). Qual è la relazione tra queste due
 unità di misura?
   \begin{keyans}
     \item* $\SI{1}{\angstrom}=\SI{1e5}{fm}$.
     \item $\SI{1}{\angstrom}=\SI{1e-5}{fm}$.
     \item $\SI{1}{\angstrom}=\SI{1e-15}{fm}$.
     \item $\SI{1}{\angstrom}=\SI{1e3}{fm}$.
   \end{keyans}
\end{enumext}

\printkeyans[columns=4]{sabastiano2}
\end{document}
